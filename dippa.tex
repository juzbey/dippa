% Lines starting with a percent sign (%) are comments. LaTeX will
% not process those lines. Similarly, everything after a percent
% sign in a line is considered a comment. To produce a percent sign
% in the output, write \% (backslash followed by the percent sign).
% ==================================================================
% Usage instructions:
% ------------------------------------------------------------------
% The file is heavily commented so that you know what the various
% commands do. Feel free to remove any comments you don't need from
% your own copy. When redistributing the example thesis file, please
% retain all the comments for the benefit of other thesis writers!
% ==================================================================
% Compilation instructions:
% ------------------------------------------------------------------
% Use pdflatex to compile! Input images are expected as PDF files.
% Example compilation:
% ------------------------------------------------------------------
% > pdflatex thesis-example.tex
% > bibtex thesis-example
% > pdflatex thesis-example.tex
% > pdflatex thesis-example.tex
% ------------------------------------------------------------------
% You need to run pdflatex multiple times so that all the cross-references
% are fixed. pdflatex will tell you if you need to re-run it (a warning
% will be issued)
% ------------------------------------------------------------------
% Compilation has been tested to work in ukk.cs.hut.fi and kosh.hut.fi
% - if you have problems of missing .sty -files, then the local LaTeX
% environment does not have all the required packages installed.
% For example, when compiling in vipunen.hut.fi, you get an error that
% tikz.sty is missing - in this case you must either compile somewhere
% else, or you cannot use TikZ graphics in your thesis and must therefore
% remove or comment out the tikz package and all the tikz definitions.
% ------------------------------------------------------------------

% General information
% ==================================================================
% Package documentation:
%
% The comments often refer to package documentation. (Almost) all LaTeX
% packages have documentation accompanying them, so you can read the
% package documentation for further information. When a package 'xxx' is
% installed to your local LaTeX environment (the document compiles
% when you have \usepackage{xxx} and LaTeX does not complain), you can
% find the documentation somewhere in the local LaTeX texmf directory
% hierarchy. In ukk.cs.hut.fi, this is /usr/texlive/2008/texmf-dist,
% and the documentation for the titlesec package (for example) can be
% found at /usr/texlive/2008/texmf-dist/doc/latex/titlesec/titlesec.pdf.
% Most often the documentation is located as a PDF file in
% /usr/texlive/2008/texmf-dist/doc/latex/xxx, where xxx is the package name;
% however, documentation for TikZ is in
% /usr/texlive/2008/texmf-dist/doc/latex/generic/pgf/pgfmanual.pdf
% (this is because TikZ is a front-end for PGF, which is meant to be a
% generic portable graphics format for LaTeX).
% You can try to look for the package manual using the ``find'' shell
% command in Linux machines; the find databases are up-to-date at least
% in ukk.cs.hut.fi. Just type ``find xxx'', where xxx is the package
% name, and you should find a documentation file.
% Note that in some packages, the documentation is in the DVI file
% format. In this case, you can copy the DVI file to your home directory,
% and convert it to PDF with the dvipdfm command (or you can read the
% DVI file directly with a DVI viewer).
%
% If you can't find the documentation for a package, just try Googling
% for ``latex packagename''; most often you can get a direct link to the
% package manual in PDF format.
% ------------------------------------------------------------------


% Document class for the thesis is report
% ------------------------------------------------------------------
% You can change this but do so at your own risk - it may break other things.
% Note that the option pdftext is used for pdflatex; there is no
% pdflatex option.
% ------------------------------------------------------------------
\documentclass[12pt,a4paper,oneside,pdftex]{report}

% The input files (tex files) are encoded with the latin-1 encoding
% (ISO-8859-1 works). Change the latin1-option if you use UTF8
% (at some point LaTeX did not work with UTF8, but I'm not sure
% what the current situation is)
\usepackage[utf8]{inputenc}
% OT1 font encoding seems to work better than T1. Check the rendered
% PDF file to see if the fonts are encoded properly as vectors (instead
% of rendered bitmaps). You can do this by zooming very close to any letter
% - if the letter is shown pixelated, you should change this setting
% (try commenting out the entire line, for example).
\usepackage[OT1]{fontenc}
% The babel package provides hyphenating instructions for LaTeX. Give
% the languages you wish to use in your thesis as options to the babel
% package (as shown below). You can remove any language you are not
% going to use.
% Examples of valid language codes: english (or USenglish), british,
% finnish, swedish; and so on.
\usepackage[finnish,swedish,english]{babel}


% Font selection
% ------------------------------------------------------------------
% The default LaTeX font is a very good font for rendering your
% thesis. It is a very professional font, which will always be
% accepted.
% If you, however, wish to spicen up your thesis, you can try out
% these font variants by uncommenting one of the following lines
% (or by finding another font package). The fonts shown here are
% all fonts that you could use in your thesis (not too silly).
% Changing the font causes the layouts to shift a bit; you many
% need to manually adjust some layouts. Check the warning messages
% LaTeX gives you.
% ------------------------------------------------------------------
% To find another font, check out the font catalogue from
% http://www.tug.dk/FontCatalogue/mathfonts.html
% This link points to the list of fonts that support maths, but
% that's a fairly important point for master's theses.
% ------------------------------------------------------------------
% <rant>
% Remember, there is no excuse to use Comic Sans, ever, in any
% situation! (Well, maybe in speech bubbles in comics, but there
% are better options for those too)
% </rant>

% \usepackage{palatino}
% \usepackage{tgpagella}



% Optional packages
% ------------------------------------------------------------------
% Select those packages that you need for your thesis. You may delete
% or comment the rest.

% Natbib allows you to select the format of the bibliography references.
% The first example uses numbered citations:
\usepackage[square,sort&compress,numbers]{natbib}
% The second example uses author-year citations.
% If you use author-year citations, change the bibliography style (below);
% acm style does not work with author-year citations.
% Also, you should use \citet (cite in text) when you wish to refer
% to the author directly (\citet{blaablaa} said blaa blaa), and
% \citep when you wish to refer similarly than with numbered citations
% (It has been said that blaa blaa~\citep{blaablaa}).
% \usepackage[square]{natbib}

% The alltt package provides an all-teletype environment that acts
% like verbatim but you can use LaTeX commands in it. Uncomment if
% you want to use this environment.
% \usepackage{alltt}

% The eurosym package provides a euro symbol. Use with \euro{}
\usepackage{eurosym}

% Verbatim provides a standard teletype environment that renderes
% the text exactly as written in the tex file. Useful for code
% snippets (although you can also use the listings package to get
% automatic code formatting).
\usepackage{verbatim}

% The listing package provides automatic code formatting utilities
% so that you can copy-paste code examples and have them rendered
% nicely. See the package documentation for details.
% \usepackage{listings}

% The fancuvrb package provides fancier verbatim environments
% (you can, for example, put borders around the verbatim text area
% and so on). See package for details.
% \usepackage{fancyvrb}

% Supertabular provides a tabular environment that can span multiple
% pages.
%\usepackage{supertabular}
% Longtable provides a tabular environment that can span multiple
% pages. This is used in the example acronyms file.
\usepackage{longtable}

% The fancyhdr package allows you to set your the page headers
% manually, and allows you to add separator lines and so on.
% Check the package documentation.
% \usepackage{fancyhdr}

% Subfigure package allows you to use subfigures (i.e. many subfigures
% within one figure environment). These can have different labels and
% they are numbered automatically. Check the package documentation.
\usepackage{subfigure}

% The titlesec package can be used to alter the look of the titles
% of sections, chapters, and so on. This example uses the ``medium''
% package option which sets the titles to a medium size, making them
% a bit smaller than what is the default. You can fine-tune the
% title fonts and sizes by using the package options. See the package
% documentation.
\usepackage[medium]{titlesec}

% The TikZ package allows you to create professional technical figures.
% The learning curve is quite steep, but it is definitely worth it if
% you wish to have really good-looking technical figures.
\usepackage{tikz}
% You also need to specify which TikZ libraries you use
\usetikzlibrary{positioning}
\usetikzlibrary{calc}
\usetikzlibrary{arrows}
\usetikzlibrary{decorations.pathmorphing,decorations.markings}
\usetikzlibrary{shapes}
\usetikzlibrary{patterns}


% The aalto-thesis package provides typesetting instructions for the
% standard master's thesis parts (abstracts, front page, and so on)
% Load this package second-to-last, just before the hyperref package.
% Options that you can use:
%   mydraft - renders the thesis in draft mode.
%             Do not use for the final version.
%   doublenumbering - [optional] number the first pages of the thesis
%                     with roman numerals (i, ii, iii, ...); and start
%                     arabic numbering (1, 2, 3, ...) only on the
%                     first page of the first chapter
%   twoinstructors  - changes the title of instructors to plural form
%   twosupervisors  - changes the title of supervisors to plural form
\usepackage[mydraft,twosupervisors]{aalto-thesis}
%\usepackage[mydraft,doublenumbering]{aalto-thesis}
%\usepackage{aalto-thesis}


% Hyperref
% ------------------------------------------------------------------
% Hyperref creates links from URLs, for references, and creates a
% TOC in the PDF file.
% This package must be the last one you include, because it has
% compatibility issues with many other packages and it fixes
% those issues when it is loaded.
\RequirePackage[pdftex]{hyperref}
% Setup hyperref so that links are clickable but do not look
% different
\hypersetup{colorlinks=false,raiselinks=false,breaklinks=true}
\hypersetup{pdfborder={0 0 0}}
\hypersetup{bookmarksnumbered=true}
% The following line suggests the PDF reader that it should show the
% first level of bookmarks opened in the hierarchical bookmark view.
\hypersetup{bookmarksopen=true,bookmarksopenlevel=1}
% Hyperref can also set up the PDF metadata fields. These are
% set a bit later on, after the thesis setup.


% Thesis setup
% ==================================================================
% Change these to fit your own thesis.
% \COMMAND always refers to the English version;
% \FCOMMAND refers to the Finnish version; and
% \SCOMMAND refers to the Swedish version.
% You may comment/remove those language variants that you do not use
% (but then you must not include the abstracts for that language)
% ------------------------------------------------------------------
% If you do not find the command for a text that is shown in the cover page or
% in the abstract texts, check the aalto-thesis.sty file and locate the text
% from there.
% All the texts are configured in language-specific blocks (lots of commands
% that look like this: \renewcommand{\ATCITY}{Espoo}.
% You can just fix the texts there. Just remember to check all the language
% variants you use (they are all there in the same place).
% ------------------------------------------------------------------
\newcommand{\TITLE}{Implementation of rolling forecast and planning system – Implications for the management system}
\newcommand{\FTITLE}{Rullaavan suunnittelun käyttöönotto ja haasteet johtamisjärjestelmän kannalta}
\newcommand{\STITLE}{Den stora stygga vargen:}
%\newcommand{\SUBTITLE}{Re-inventing the Wheel}
%\newcommand{\FSUBTITLE}{Uusi organisaatio, uudet pyörät}
%\newcommand{\SSUBTITLE}{Lilla Vargens universum}
\newcommand{\DATE}{February 20, 2013}
\newcommand{\FDATE}{20.2.2013}
\newcommand{\SDATE}{Den 18 Juni 2011}

% Supervisors and instructors
% ------------------------------------------------------------------
% If you have two supervisors, write both names here, separate them with a
% double-backslash (see below for an example)
% Also remember to add the package option ``twosupervisors'' or
% ``twoinstructors'' to the aalto-thesis package so that the titles are in
% plural.
% Example of one supervisor:
%\newcommand{\SUPERVISOR}{Professor Antti Ylä-Jääski}
%\newcommand{\FSUPERVISOR}{Professori Antti Ylä-Jääski}
%\newcommand{\SSUPERVISOR}{Professor Antti Ylä-Jääski}
% Example of twosupervisors:
\newcommand{\SUPERVISOR}{Professor Ilkka Kauranen}
\newcommand{\FSUPERVISOR}{Professori Ilkka Kauranen}
\newcommand{\SSUPERVISOR}{Professor Antti Ylä-Jääski\\
  Professor Pekka Perustieteilijä}

% If you have only one instructor, just write one name here
\newcommand{\INSTRUCTOR}{Master of Economics Jarno Jahnukainen}
\newcommand{\FINSTRUCTOR}{Taloustieteiden maisteri Jarno Jahnukainen}
\newcommand{\SINSTRUCTOR}{Diplomingenjör Olli Ohjaaja}
% If you have two instructors, separate them with \\ to create linefeeds
% \newcommand{\INSTRUCTOR}{Olli Ohjaaja M.Sc. (Tech.)\\
%  Elli Opas M.Sc. (Tech)}
%\newcommand{\FINSTRUCTOR}{Diplomi-insinööri Olli Ohjaaja\\
%  Diplomi-insinööri Elli Opas}
%\newcommand{\SINSTRUCTOR}{Diplomingenjör Olli Ohjaaja\\
%  Diplomingenjör Elli Opas}

% If you have two supervisors, it is common to write the schools
% of the supervisors in the cover page. If the following command is defined,
% then the supervisor names shown here are printed in the cover page. Otherwise,
% the supervisor names defined above are used.
%\newcommand{\COVERSUPERVISOR}{Professor Ilkka Kauranen, Aalto University}

% The same option is for the instructors, if you have multiple instructors.
% \newcommand{\COVERINSTRUCTOR}{Olli Ohjaaja M.Sc. (Tech.), Aalto University\\
%  Elli Opas M.Sc. (Tech), Aalto SCI}


% Other stuff
% ------------------------------------------------------------------
\newcommand{\PROFESSORSHIP}{Knowledge intensive business}
\newcommand{\FPROFESSORSHIP}{Tietointensiivinen liiketoiminta}
\newcommand{\SPROFESSORSHIP}{Datakommunikationsprogram}
% Professorship code is the same in all languages
\newcommand{\PROFCODE}{T3008}
\newcommand{\KEYWORDS}{budgeting, forecasting, rolling forecasting, planning, management system, management control system, management controls}
\newcommand{\FKEYWORDS}{budjetointi, ennustaminen, rullaava suunnittelu, johtamisjärjestelmä, johdon ohjausjärjestelmä, johdon ohjauskeinot}
\newcommand{\SKEYWORDS}{omsättning, kassaflöde, värdepappersmarknadslagen,
yrkesutövare, intresseföretag, verifieringskedja}
\newcommand{\LANGUAGE}{Finnish}
\newcommand{\FLANGUAGE}{Suomi}
\newcommand{\SLANGUAGE}{Engelska}

% Author is the same for all languages
\newcommand{\AUTHOR}{Jussi Juvonen}


% Currently the English versions are used for the PDF file metadata
% Set the PDF title
\hypersetup{pdftitle={\TITLE\ \SUBTITLE}}
% Set the PDF author
\hypersetup{pdfauthor={\AUTHOR}}
% Set the PDF keywords
\hypersetup{pdfkeywords={\KEYWORDS}}
% Set the PDF subject
\hypersetup{pdfsubject={Master's Thesis}}


% Layout settings
% ------------------------------------------------------------------

% When you write in English, you should use the standard LaTeX
% paragraph formatting: paragraphs are indented, and there is no
% space between paragraphs.
% When writing in Finnish, we often use no indentation in the
% beginning of the paragraph, and there is some space between the
% paragraphs.

% If you write your thesis Finnish, uncomment these lines; if
% you write in English, leave these lines commented!
\setlength{\parindent}{0pt}
\setlength{\parskip}{1ex}

% Use this to control how much space there is between each line of text.
% 1 is normal (no extra space), 1.3 is about one-half more space, and
% 1.6 is about double line spacing.
% \linespread{1} % This is the default
% \linespread{1.3}

% Bibliography style
% acm style gives you a basic reference style. It works only with numbered
% references.
\bibliographystyle{acm}
% Plainnat is a plain style that works with both numbered and name citations.
% \bibliographystyle{plainnat}


% Extra hyphenation settings
% ------------------------------------------------------------------
% You can list here all the files that are not hyphenated correctly.
% You can provide many \hyphenation commands and/or separate each word
% with a space inside a single command. Put hyphens in the places where
% a word can be hyphenated.
% Note that (by default) LaTeX will not hyphenate words that already
% have a hyphen in them (for example, if you write ``structure-modification
% operation'', the word structure-modification will never be hyphenated).
% You need a special package to hyphenate those words.
\hyphenation{joh-ta-mis-jär-jes-tel-mä joh-ta-mis-jär-jes-tel-män oh-ja-us-jär-jes-tel-mi-en käyt-tö-me-ka-nis-mi-en ym-mär-ret-tä-vä käy-tän-tö-läh-töi-seen}

% The preamble ends here, and the document begins.
% Place all formatting commands and such before this line.
% ------------------------------------------------------------------
\begin{document}
% This command adds a PDF bookmark to the cover page. You may leave
% it out if you don't like it...
\pdfbookmark[0]{Cover page}{bookmark.0.cover}
% This command is defined in aalto-thesis.sty. It controls the page
% numbering based on whether the doublenumbering option is specified
\startcoverpage

% Cover page
% ------------------------------------------------------------------
% Options: finnish, english, and swedish
% These control in which language the cover-page information is shown
\coverpage{finnish}

% Abstracts
% ------------------------------------------------------------------
% Include an abstract in the language that the thesis is written in,
% and if your native language is Finnish or Swedish, one in that language.


% Abstract in Finnish
% ------------------------------------------------------------------
\thesisabstract{finnish}{
Tämän tutkimuksen kohteena on kohdeyrityksessä tapahtuva rullaavan suunnitteluprosessin käyttöönotto. Työssä keskitytään erityisesti rullaavan suunnitteluprosessin ja johtamisjärjestelmän suhteeseen. Työn tavoitteena on tarjota konkreettisia suosituksia rullaavaan suunnitteluun siirtymisen edistämiseksi kohdeyrityksessä. Työssä käytetyt tutkimusmenetelmät ovat kirjallisuustutkimus ja haastattelut. Kirjallisuuden pohjalta muodostetaan ensiksi käsitys rullaavan suunnittelun ja johtamisjärjestelmän suhteesta sekä rullaavan suunnittelun vaikutuksista yrityksen toimintaan. Empiirisessä osassa perehdytään kohdeyrityksen suunnitteluprosessiin  sekä sisäisen materiaalin avulla että haastattelututkimuksen keinoin. Samalla selvitettiin kohdeyrityksessä vallitsevia käsityksiä liittyen rullaavan suunnittelun käyttöönottoon ja sen vaikutuksista johtamisjärjestelmään. Työn tulokset perustuvat sekä kirjallisuuden että haastatteluaineiston analysointiin ja ne esitetään kohdeyrityksen rullaavan suunnitteluprosessin käyttöönottoa edistävien suositusten muodossa. Tärkeimmät suositukset pitävät sisällään muun muassa seuraavia seikkoja: Rullaavan suunnittelun ja johtamisjärjestelmän suunnitelmallisen yhteensovittamisen jatkaminen. Rullaavan suunnittelun etujen selvittäminen eri osapuolille entistä painokkaammin. Rullaavan suunnitteluprosessin kulun selvittäminen eri osapuolille entistä havainnollisemmin. Palkitsemiskäytäntöjen kehittäminen rullaavaa suunnitteluprosessia tukeviksi. Kannustimien lisääminen rullaavan suunnittelun tekemiseksi  osaksi päivittäistä johtamistyötä. Rullaavaan suunnitteluun liittyvän raportoinnin kehittäminen entistä joustavammaksi. Rullaavaan suunnitteluun liittyvän tavoiteasetannan ja tuloskorttien tavoitteiden suhteen tarkistaminen. Palautteen keräämisen ja siihen reagoinnin kehittäminen yhä systemaattisemmaksi. Annetut suositukset tiivistävät kohdeyrityksessä vallitsevan käsityksen ja toteuttamalla niitä voidaan edistää rullaavan suunnittelun käyttöönottoa.}

% Abstract in English
% ------------------------------------------------------------------
\thesisabstract{english}{
This study focuses on the implementation of the rolling forecast and planning process of the case company.  The main emphasis is on the relationship between the rolling planning process and the management system of the company. The objective of this study is to of-fer concrete recommendations in order to facilitate the implementation of the rolling plan-ning in the case company. The methodologies used in this study include a literature study accompanied by the empirical interviews. First basing on the literature, the relationship between rolling planning and management systems is conceptualized. Also the main char-acteristics of rolling forecasting and planning and their use are studied. In the empirical part of the study, internal documentation is used to describe the new planning process and interviews were conducted in order to capture the management’s current understanding of the subject. Particularly, the interviews focused on the outlook for the implementation of the rolling forecast and planning process and its implications for the management system.

The findings of this study are based on a rigorous analysis of the literature and the data acquired from the interviews. As findings, a set of concrete recommendations are offered in order to facilitate the implementation of rolling planning process. The most important recommendations are as follows: Continuing the systematic integration of the rolling planning process and the management system. Bringing up the benefits of the rolling planning even more emphatically than before. Clarifying the rolling planning process for different parties. Developing reward and compensation methods that support rolling plan-ning process. Boosting the intensives for making the rolling planning as a part of the daily routine. Developing even more flexible reporting possibilities. Adjusting the relationship of the target setting between rolling planning and balanced scorecards. The given recom-mendations encapsulate the current collective understanding in the case company. By fol-lowing these, the implementation of the rolling forecast and planning process can be fa-cilitated.
}

% Acknowledgements
% ------------------------------------------------------------------
% Select the language you use in your acknowledgements
\selectlanguage{finnish}

% Uncomment this line if you wish acknoledgements to appear in the
% table of contents
%\addcontentsline{toc}{chapter}{Acknowledgements}

% The star means that the chapter isn't numbered and does not
% show up in the TOC
\chapter*{Esipuhe}

Diplomityö on varmasti suurin ponnistus jokaisen teekkarin opiskeluaikana. Se on kuitenkin myös lähes varmuudella se opettavaisin. Prosessin aikana opin paljon uutta. Tiedän paljon enemmän budjetoinnista ja rullaavasta suunnittelusta kuin aikaisemmin, sain kokemusta haastattelemisesta ja osaan kirjoittaa siistin tutkimusraportin. Lisäksi opin, että apuakin on välillä kysyttävä. Se on osa prosessia, ja vain siten voidaan päästä parhaisiin tuloksiin.

Olen kiitollinen SOK:lle, joka tarjosi tilaisuuden tämän diplomityön tekemiselle. Erityisesti kiitän talousjohtajana SOK:lla toimivaa ohjaajaani TTM Jarno Jahnukaista, joka tuki minua työssäni aina loppuun asti. Kiitän myös haastateltavia, jotka ystävällisesti tarjosivat aikaansa ja näkemystään tätä työtä varten. Suuri kiitos valvojalleni professori Ilkka Kauraselle arvokkaasta näkemyksestä ja avusta työn kokonaisuuden muokkaamisessa. Kiitos läheisilleni, jotka ovat antaneet työrauhan ja uskoneet valmistumiseeni, vaikkakin hitaasti mutta varmasti.

Toivon että työn tuloksista on hyötyä S-ryhmän suunnitteluprosessin ja johtamisjärjestelmän kehittämissä. Omana tavoitteenani on työn tekemisen aikana ollut kehittää tutkimustaitojani, projektin koordinointikykyä ja sisäistää monimutkainen kokonaisuus, jonka yrityksen johtamisjärjestelmä, ennustamis- ja suunnitteluprosessi muodostavat. Uskonkin, että tämän diplomityö tuki ammatillista kasvuani monin eri tavoin. Toivotan kiintoisia hetkiä työni parissa!

\vskip 10mm

\noindent Espoossa \FDATE
\vskip 5mm
\noindent\AUTHOR

% Acronyms
% ------------------------------------------------------------------
% Use \cleardoublepage so that IF two-sided printing is used
% (which is not often for masters theses), then the pages will still
% start correctly on the right-hand side.
\cleardoublepage
% Example acronyms are placed in a separate file, acronyms.tex
% \input{acronyms}

% Table of contents
% ------------------------------------------------------------------
\cleardoublepage
% This command adds a PDF bookmark that links to the contents.
% You can use \addcontentsline{} as well, but that also adds contents
% entry to the table of contents, which is kind of redundant.
% The text ``Contents'' is shown in the PDF bookmark.
\pdfbookmark[0]{Contents}{bookmark.0.contents}
\tableofcontents

% List of tables
% ------------------------------------------------------------------
% You only need a list of tables for your thesis if you have very
% many tables. If you do, uncomment the following two lines.
% \cleardoublepage
% \listoftables

% Table of figures
% ------------------------------------------------------------------
% You only need a list of figures for your thesis if you have very
% many figures. If you do, uncomment the following two lines.
% \cleardoublepage
% \listoffigures

% The following label is used for counting the prelude pages
\label{pages-prelude}
\cleardoublepage

%%%%%%%%%%%%%%%%% The main content starts here %%%%%%%%%%%%%%%%%%%%%
% ------------------------------------------------------------------
% This command is defined in aalto-thesis.sty. It controls the page
% numbering based on whether the doublenumbering option is specified
\startfirstchapter

% Add headings to pages (the chapter title is shown)
\pagestyle{headings}

% The contents of the thesis are separated to their own files.
% Edit the content in these files, rename them as necessary.
% ------------------------------------------------------------------

% \input{1introduction.tex}

\chapter{Johdanto}
\label{chapter:johdanto}

\section{Motivaatio}

Yritysten on nykyaikana seurattava tiiviisti ympäristöään ja ympäröivän maailman tapahtumia. Vaikka yrityksen strategia olisi huippuunsa hiottu, ei ympäristön muutoksia pystytä aina näkemään ennalta. Muutoksiin voidaan kuitenkin pyrkiä varautumaan ennustamisen avulla. Tulevaisuuden näkymien ennustaminen on tärkeää toiminnan suunnittelun kannalta, mistä johtuen sille tulisi myös osoittaa riittävästi resursseja (Clarke 2007). Toisaalta on syytä huomata, että kyky ennustaa virheettömästi ei vielä riitä, vaan reagointikyky ja suunnitelmien päivittäminen sekä toimeenpano on vähintään yhtä tärkeää (Barrett & Hope 2006).

Toimintaympäristön muuttuminen edellyttää sopeutumista uusiin olosuhteisiin, minkä pitäisi tapahtua ensisijaisesti uusia, päivitettyjä strategioita toteuttamalla (Apanaschik 2007). Tällöin muutosta pystytään hallitsemaan ja ohjaamaan halutulla tavalla eteenpäin. Jotta muutosta voitaisiin johtaa strategian kautta, on ennustaminen ja lyhyen tähtäimen suunnittelu tärkeää kytkeä strategiaan ja pidemmän aikavälin tavoitteisiin (Lamoreaux 2011). Tämä mahdollistaa joustavan päätöksenteon tilanteen mukaisesti, auttaen samalla huomioimaan strategiassa asetetut suuntaviivat. Koska lyhytaikaisen toiminnan suunnittelun lähtökohtana pitäisi hyödyntää sekä ennusteita että yrityksen strategiaa, on suunnitteluprosessissa huomioitava riittävällä tasolla molemmat.

Epävakaa liiketoimintaympäristö ei ole ainoa haaste suunnitteluprosessille. Yritysten on pystyttävä myös raportoimaan odotettua taloudellista tulosta säännöllisesti. (Apanaschik 2007) Lisäksi internetin merkitys on noussut keskeiseksi. Yhä useammin liiketoimintaympäristöä dominoi verkkokauppa ja sähköinen kaupankäynti, jolloin taloudellisen johtamisen tekniikoiden on pystyttävä vastaamaan sähköisen kaupankäynnin vaatimaan nopeuteen ja joustavuuteen (Fanning 2000). Yritysten suunnitteluprosessilta ja -järjestelmiltä vaaditaan siis koko ajan enemmän.

Perinteisesti lyhyen aikavälin suunnitteluun ja osittain myös ennustamisen tarkoituksiin, on käytetty vuosibudjetointia. Viime aikoina budjetointi on kuitenkin saanut osakseen kasvavaa kritiikkiä, ja monet näkevät sen kankeaksi ja byrokraattiseksi jäänteeksi menneisyydestä. Budjetointiin käytetään myös huomattavia määriä aikaa, josta suuri osa kuluu datan keräämiseen ja hallintaan (Apanaschik 2007). Yhä useammat yritykset ovatkin lähteneet miettimään, kuinka ennustamis- ja suunnitteluprosessia – useimmissa yrityksissä käytännössä budjetointiprosessia – voitaisiin kehittää paremmin nykyajan vaatimuksia vastaavaksi.

Tässä diplomityössä tutkitaan rullaavaa suunnittelua ja sen käyttöönottoa. Rullaava suunnittelu pitää sisällään taloudellisten ennusteiden ja operatiivisten suunnitelmien päivittämisen, sekä tarpeen tullen strategisten vaihtoehtojen ja suunnitelmien tarkastelun. Tämän tutkimuksen kohdeyrityksessä S-ryhmässä ollaan korvaamassa pitkään käytössä olleita perinteisiä budjetointimenetelmiä rullaavalla suunnittelulla. Tästä johtuen on tärkeää ymmärtää sekä perinteisen budjetoinnin lainalaisuudet että rullaavan suunnitteluprosessin keskeiset ominaisuudet. Budjetointia tulisi tutkia osana johtamisjärjestelmää ja sen eri alueista ja toiminnoista muodostuvaa kokonaisuutta (Hansen et al. 2003). Sama pätee rullaavaan suunnitteluun, sillä budjetoinnin lailla myös rullaava suunnittelu toimii yhtenä johtamisjärjestelmän komponenttina. Jotta uudesta suunnitteluprosessista ja rullaavan suunnittelun järjestelmästä saataisiin mahdollisimman suuri hyöty, on sen suhde yrityksen johtamisjärjestelmään ja sen eri alueisiin määriteltävä ja ymmärrettävä.

\section{Ongelman määritys}
\label{section:ongelma}

S-ryhmä koostuu 20 alueosuuskaupasta, joista kukin toimii omana itsenäisenä yrityksenään. Alueosuuskauppojen toimintaa koordinoi Suomen Osuuskauppojen Keskuskunta (SOK), joka tuottaa osuuskaupoille hankinta-, asiantuntija- ja tukipalveluita ja vastaa S-ryhmän strategisesta ohjauksesta ja sen eri ketjujen kehittämisestä. S-ryhmä toimii vahvasti ketjujen kautta, minkä ansiosta keskitetty hankinta voidaan yhdistää osuuskauppojen paikallistuntemukseen. Jokaista ketjua ohjaa oma ketjuohjausorganisaatio, joka vastaa oman toimialansa ketjutoiminnan koordinoinnista, kehittämisestä, ohjaamisesta ja valvomisesta. Ketjuohjauksen osuus liiketoiminnan suunnittelussa on alueosuuskaupoille oleellinen. Jotta tätä kokonaisuutta voitaisiin hallita entistä tehokkaammin, tarvitaan yhtenäisiä, koko S-ryhmän kattavia tietojärjestelmiä.

Viime vuosina S-ryhmässä on ollut käynnissä rullaavan suunnitteluprosessin Supron käyttöönottoprojekti, joka on hyvä esimerkki S-ryhmän prosesseja sekä järjestelmiä yhtenäistävästä hankkeesta. Projektiin liittyy sekä suunnitteluprosessin kehittäminen kokonaisuutena että siihen saumattomasti integroituvan ja sitä tukevan suunnittelujärjestelmän käyttöönotto. Kehittämistyön tavoitteena on parantaa koko S ryhmän reagointikykyä nopeasti muuttuvassa liiketoimintaympäristössä. Samalla yhtenäistyvät myös johtamiskäytännöt ja johtamisen työkalut, minkä odotetaan johtavan parempaan johtamiseen ja suunnitelmien toteuttamiseen. Ensisijainen haaste projektin onnistumisessa suunnittelujärjestelmän käyttöönoton ohella on uudistetun suunnitteluprosessin koordinointi ja sovittaminen S-ryhmän johtamisjärjestelmään. Näistä lähtökohdista syntyi tilaisuus myös tämän diplomityön tekemiselle. Koska suunnitteluprosessi kytkeytyy hyvin vahvasti johtamiseen, on johtamisen prosessia ja käytäntöjä tarkasteltava uudelleen. Toisaalta kysymys on uuden ajattelutavan omaksumisesta, jonka rullaava suunnittelu tuo osaksi päivittäistä toimintaa ja johtamista.

Rullaavia konsepteja käsittelevä kirjallisuus on suhteellisen kirjavaa, ja perustuu suureksi osaksi tutkimuslaitosten ja konsultointiyritysten käytäntölähtöiseen tutkimukseen. Tämän takia tieto on pirstaloitunutta ja sen saatavuus vaihtelee. Hillestadin (2011) mukaan rullaavan suunnittelun kaltaisten uusien menetelmien käyttöönottamisesta yrityksissä on tehty vain vähän empiirisiä tutkimuksia. Lisäksi rullaava suunnittelu ja esimerkiksi tuloskortit ovat läheisesti yhteydessä yrityksen strategiaan ja johtamisjärjestelmään, mutta tämänkaltaisten vaihtoehtoisten ohjausjärjestelmien käyttöönottamisen vaikutuksista tiedetään käytännössä vain vähän (Østergren & Stensaker 2011). Tämä tutkimus tuo näin ollen mielenkiintoisen ja tuoreen näkökulman aiheeseen.

\section{Tavoitteet ja aiheen rajaus}
\label{section:tavoitteet}

Tämän diplomityön tavoitteena on tarjota kohdeyritykselle konkreettisia suosituksia rullaavan suunnitteluprosessin käyttöönoton edistämiseksi. Rullaavan suunnittelun käyttöönotossa olennainen haaste on uuden suunnitteluprosessin sovittaminen S-ryhmän johtamisjärjestelmään. Tästä johtuen työssä perehdytään johtamisjärjestelmän ja johtamisprosessin käsitteisiin. Koska perinteisestä budjetoinnista on tarkoitus luopua, tutustutaan työssä myös budjetoinnin ominaispiirteisiin ja käyttötarkoituksiin lähtötilanteen ymmärtämiseksi. Lisäksi työssä esitellään perinteisen budjetoinnin keskeiset ongelmat ja vaihtoehtoisia menetelmiä budjetoinnille. Tämän jälkeen perehdytään rullaavan suunnittelun ominaispiirteisiin ja siihen, miten ennustaminen ja rullaava suunnittelu kytkeytyvät osaksi yrityksen johtamisjärjestelmää.

Työn empiirisessä osassa selvitettiin teoriaosan tueksi haastatteluin eri osapuolten näkemyksiä liittyen suunnitteluprosessin uudistamiseen. Tutkimusta varten haastateltiin useita eri liiketoiminnan johtajia sekä suunnittelijoita eri alueosuuskaupoista, kuin myös ketjuohjauksesta vastaavia henkilöitä SOK:lta. Haastatteluiden avulla haluttiin selvittää, minkälaisia käsityksiä eri henkilöillä on rullaavasta suunnitteluprosessista ja miten sen käyttöönoton koetaan vaikuttavan johtamisjärjestelmään. Samalla tiedusteltiin, mitkä asiat vielä koetaan haasteelliseksi ja mihin suuntaan uuden suunnitteluprosessin käyttöönottoa pitäisi ohjata. S-ryhmässä vallitsevaa kollektiivista käsitystä haluttiin selvittää erityisesti seuraavia kysymyksiä koskien:

\begin{itemize}
% You can use this command to set the items in the list closer to each other
% (ITEM SEParation, the vertical space between the list items)
\setlength{\itemsep}{0pt}
\item Millä eri keinoilla voitaisiin edistää kokonaisuuden ohjaamista ja johtamista uudessa suunnitteluprosessissa?
\item Millaiset suunnitteluprosessiin liittyvät käytännöt voisivat parhaiten tukea yhtenäisen, rullaavan suunnitteluprosessin käyttöönottoa S-ryhmässä?
\item Mihin asioihin tulisi kiinnittää erityistä huomiota, jotta uudesta suunnitteluprosessista saataisiin täysi hyöty mahdollisimman nopeasti?
\end{itemize}

Rullaavan suunnittelun käyttöönottoa käsitellään työssä johtamisen ja johtamisjärjestelmän näkökulmasta. Esimerkiksi järjestelmän käyttöönottoon ja käytettävyyteen liittyvät yksityiskohdat on rajattu käsittelyn ulkopuolelle. Työssä käsitellään melko kattavasti eri suunnitteluprosessiin kytköksissä olevia asioita, sillä ne kuuluvat olennaisina osina johtamisjärjestelmään. Yksittäisiin johtamisjärjestelmän komponentteihin ei kuitenkaan syvennytä tarkemmin, itse suunnitteluprosessia lukuun ottamatta.

Työn johdantoa seuraa kirjallisuuskatsaus (lisätietoa liitteessä 1), jossa luodaan työn teoreettinen pohja. Työn empiirisessä osiossa esitellään lyhyesti S-ryhmää organisaationa ja sen suunnitteluprosessia, minkä jälkeen kuvaillaan tutkimusmenetelmät ja haastattelujen tulokset. Työn viimeisessä vaiheessa muodostetaan synteesi teorian ja empirian pohjalta ja päätetään työ johtopäätöksiin ja yhteenvetoon. Työn rakenne esitetään alla kuvassa 1.

\begin{figure}[ht]
  \begin{center}
    \includegraphics[width=\textwidth]{rakenne.png}
    \caption{Tutkimuksen rakenne.}
    \label{fig:rakenne}
  \end{center}
\end{figure}

\section{Keskeiset käsitteet}

\textbf{Johtamisjärjestelmä}

Yrityksen johtamisjärjestelmä on laaja kokonaisuus, joka koostuu yrityksen prosesseista, käytännöistä ja työkaluista. Johtamisjärjestelmä kattaa eri työkalut ja käytännöt aina strategian luomisesta operatiiviseen toimintaan. Curtis et al. (2002) määrittelevät johtamisjärjestelmän johtamisprosessien, -käytäntöjen, -työkalujen, -periaatteiden,  politiikkojen ja -mittareiden integroiduksi, toisiaan tukevaksi ja organisaation vakiintunutta johtamista kuvaavaksi kokonaisuudeksi (Oiva 2007).

\textbf{Budjetointi}

Budjetoinnissa luodaan tavallisesti yksityiskohtainen rahamääräinen suunnitelma sovitulle ajanjaksolle liiketoiminnan työkaluksi. Tyypillisesti uusi budjetti laaditaan kalenterivuosittain. Bhimani et al. (2008) määrittelevät budjetin määrälliseksi suunnitelmaksi, jonka johto laatii toimeenpanon ja johtamisen tueksi. Poikkeuksetta eri määritelmät mainitsevat budjetin olevan suunnitelma resurssien – kuten liikevaihto ja muut rahavirrat – hankinnasta sekä niiden käytöstä tietyn ajanjakson aikana (Etim & Agara 2011).

\textbf{Rullaava ennustaminen}

Rullaavalla ennustamisella viitataan ennusteiden säännölliseen päivittämiseen, tyypillisesti kuukauden tai vuosineljänneksen välein. Samalla ennustehorisonttia jatketaan aina siten, että siihen sisältyvien ennustejaksojen määrä säilyy vakiona. (Clarke 2007) Olennaista rullaavassa ennustamisessa on se, että ennustehorisontti on kuluvaa tilikautta pidemmällä (Åkerberg 2006). Rullaava ennustaminen on tavallisesti rajoittunut tarkistettujen taloudellisten arvioiden tuottamiseen (Fanning 2000).

\textbf{Rullaava suunnittelu}

Rullaava suunnittelu, johon kirjallisuudessa viitataan toisinaan myös rullaavana budjetointina, pitää taloudellisten ennusteiden lisäksi sisällään operatiivisten suunnitelmien päivittämisen, sekä tarpeen tullen strategisten vaihtoehtojen ja suunnitelmien tarkastelun. Rullaavaa ennustamista tehdään usein rullaavan suunnittelun alaisuudessa tapahtuvana, jolloin ne tukevat luonnollisesti toisiaan. (Fanning 2000) Kuten rullaavaa ennustamista, myös rullaavaa suunnittelua tehdään säännöllisin väliajoin kalenterivuoden kuluessa. Aina, kun uutta toteumatietoa saadaan järjestelmään, jatketaan suunnitteluhorisonttia siten, että suunnittelujaksojen määrä säilyy vakiona. Mikäli rullaavaa suunnittelua käytetään koko organisaatiossa, ei budjetoinnille välttämättä ole enää tarvetta.

\chapter{Johtamisjärjestelmä ja sen eri alueet}
\label{chapter:johtamisjärjestelmä}

\section{Johtamisjärjestelmä}

Hansen et al. (2003) toteavat, että budjetointia tulisi tutkia osana johtamisjärjestelmän ja sen eri alueiden muodostamaa kokonaisuutta. Sama pätee rullaavaan suunnitteluun, sillä budjetoinnin lailla, myös rullaava suunnittelu toimii yhtenä johtamisjärjestelmän komponenttina. Yrityksen johtamisjärjestelmä on laaja kokonaisuus, joka koostuu yrityksen prosesseista, käytännöistä ja työkaluista. Eri määritelmät painottavat hieman eri asioita, mutta niiden sisältö on olennaisin osin sama: johtamisjärjestelmä kattaa eri työkalut ja käytännöt aina strategian luomisesta operatiivisen toiminnan johtamiseen.

Salminen (2008) liittää johtamisjärjestelmän erityisesti johtamiseen. Hänen mukaansa johtamisjärjestelmään kuuluu mekanismit, joiden kautta yrityksen johto ja esimiehet johtavat yritystä. Johtamiseen kuuluu muun muassa informaation kerääminen ja analysointi, suunnittelu, tavoitteiden asettaminen, organisointi ja resursointi, motivointi, seuranta, palautteen antaminen ja ohjaus sekä henkisten ja fyysisten resurssien kehittäminen. (Salminen 2008) Lähtökohdaksi voidaan ottaa myös strategia. Kaplan ja Norton (2008) määrittelevät johtamisjärjestelmän yhtenäisiksi prosesseiksi ja välineiksi, joiden avulla strategia muodostetaan ja edelleen muutetaan operatiiviseksi toiminnaksi. Johtamisjärjestelmään kuuluu myös strategian ja operatiivisen toiminnan tehokkuuden seuraaminen ja kehittäminen. (Kaplan & Norton 2008)

Viitala ja Jylhän (2007) mukaan johtamisjärjestelmään kuuluvat muun muassa päätösvallan ja vastuun määrittely, visio- ja strategiaprosessi, tavoiteasetanta ja seurannan systematiikka, talouden suunnittelun ja seurannan järjestelmät, johdon tietojärjestelmät, palkitsemisjärjestelmät, systemaattiset palaverikäytännöt ja kehityskeskustelujärjestelmä. Curtis et al. (2002) määrittelevät johtamisjärjestelmän johtamisprosessien,  käytäntöjen, -työkalujen, -toimintaperiaatteiden, -politiikkojen ja -mittareiden integroiduksi, toisiaan tukevaksi ja organisaation vakiintunutta johtamista kuvaavaksi kokonaisuudeksi (Oiva 2007). Tämä on määritelmistä kattavin, ja tässä tutkimuksessa johtamisjärjestelmä mielletään sen mukaiseksi.

Budjetointi ja siihen pohjaava raportointi voidaan katsoa yhdeksi keskeiseksi johtamisjärjestelmän osaksi (Åkerberg 2006). Kun yrityksen suunnitteluprosessia lähdetään kehittämään, on samalla syytä tarkastella kokonaisvaltaisesti myös johtamisjärjestelmää ja johtamisprosesseja. Mikäli näin ei tehdä, törmätään käyttöönottovaiheessa todennäköisesti erinäköisiin vaikeuksiin. (Hope et al. 2007)

\section{Johdon ohjausjärjestelmät}

Johtamisjärjestelmä on itsessään niin laaja ja moniulotteinen kokonaisuus, että sitä on haastavaa tutkia jakamatta aihetta pienempiin kokonaisuuksiin. Systemaattisempi lähestymistapa aiheeseen on otettu johdon ohjausjärjestelmien (management control systems) tutkimuksessa. Johdon ohjausjärjestelmien voidaan sanoa olevan yrityksen johtamisjärjestelmän komponentteja. Esimerkiksi budjetointi ja tuloskortit voidaan luokitella tällaisiksi ohjausjärjestelmiksi (Malmi & Brown 2008).

Simonsin (1995) mukaan johdon ohjausjärjestelmät ovat muodollisia ja informaatiopohjaisia rutiineja sekä menettelytapoja, joiden avulla johto pyrkii joko pitämään organisaation toiminnan ennallaan tai muuttamaan sitä. Lisäksi hän erottaa johdon ohjausjärjestelmät päätöksenteon tukijärjestelmistä, joita ei varsinaisesti käytetä työntekijöiden ohjaukseen, vaan muihin tarkoituksiin. Merchant ja Van der Stede (2007) määrittelevät johdon ohjausjärjestelmät kokoelmaksi erilaisia johdon ohjauskeinoja, joiden avulla johto pyrkii varmistamaan työntekijöidensä yhdenmukaisen käyttäytymisen suhteessa organisaation tavoitteisiin ja strategiaan. Malmi ja Brown (2008) määrittelevät johdon ohjausmekanismeiksi kaikki välineet ja järjestelmät, joita johto käyttää varmistaakseen työntekijöiden käytöksen ja päätöksenteon olevan yhdenmukaista organisaation tavoitteiden ja strategian kanssa. 

Yksittäiset ohjausjärjestelmät voivat olla esimerkiksi perinteisiä kirjanpidon ja laskennan järjestelmiä tai hallinnollisia ohjausmekanismeja, kuten organisaatiorakenteen tai hallinnon järjestelmiä. Vastaavasti johdon ohjausjärjestelmien muodostama kokonaisuus koostuu eri ohjausmenetelmistä ja ohjausjärjestelmistä. Ohessa havainnollistus eri järjestelmien suhteesta toisiinsa Malmin ja Brownin (2008) artikkelin pohjalta (kuva 2).

Johdon ohjausjärjestelmien kokonaisuus nostettiin ensimmäisiä kertoja esille 1980-luvulla (esim. Otley 1980; Flamholtz 1983). Aihetta on tämän jälkeen tutkittu eri näkökulmista, mutta edelleen tietämys eri ohjausjärjestelmien välisistä vuorovaikutuksista on rajoittunutta (Sandelin 2008). Lisäksi vaikka erillisiä ohjausjärjestelmiä – esimerkiksi toimintolaskennan järjestelmiä, tuloskortteja tai rullaavaa ennustamista – on tutkittu melko paljon, ei uuden ohjausjärjestelmän käytön vaikutuksia laajempaan ohjausjärjestelmien kokonaisuuteen ole välttämättä näissä yhteyksissä tarkasteltu (Malmi & Brown 2008). Tutkijat ovatkin painottaneet, että ohjaamisjärjestelmien yhteistoimintaa pitäisi tutkia, jotta niiden rakennetta ja toimintaa voitaisiin paremmin ymmärtää (Fisher 1998; Otley 1999; Malmi & Brown 2008). Fisher (1998) toteaa myös, että jos eri ohjausjärjestelmien kytköksiä toisiinsa ei tunnisteta, väärien johtopäätösten riski tutkimuksessa kasvaa.

Viimeaikaisesta tutkimuksesta Malmin ja Brownin (2008) typologia huomioi viimeisimmät johdon ohjausjärjestelmiin kuuluvat kehitysaskeleet, kuten hybridijärjestelmäksi luokiteltavan tuloskortin, joka kattaa sekä taloudellisia että ei-taloudellisia näkökulmia. Typologian kehittämiseksi Malmi ja Brown (2008) analysoivat ja syntetisoivat ohjausjärjestelmien tutkimusta lähes neljän vuosikymmenen ajalta. Tuloksena syntyneen viitekehyksen (kuva 3) avulla johdon ohjausjärjestelmiä voidaan tutkia kokonaisuutena perusteellisemmalla tavalla.

Green ja Welsh (1988) määrittelevät kyberneettiset ohjausvälineet prosesseiksi, joissa verrataan suorituskykyä ennalta määritettyihin standardeihin ja tarkkaillaan samalla ilmeneviä poikkeumia, ja edelleen muokataan tämän pohjalta systeemin käyttäytymistä. Kyberneettinen systeemi luokitellaan tietojärjestelmäksi tai päätöksenteon tukijärjestelmäksi, jos johtajat itse tarkkailevat epätoivottua varianssia ja mukauttavat käyttäytymistään tai toimintaansa itsenäisesti. Mikäli käyttäytyminen kuitenkin kytketään tavoitteisiin ja vastuullisuus suorituskyvyn variansseihin, niin systeemi luokitellaan kyberneettiseksi johdon ohjausjärjestelmäksi. Kyberneettisten systeemien kirjallisuudessa esiintyvät neljä perustyyppiä ovat budjetit, taloudelliset ohjaustoimet, ei-taloudelliset ohjaustoimet ja hybridijärjestelmät. (Malmi & Brown 2008)

Aikaisempi johdon ohjausjärjestelmien tutkimus on keskittynyt vahvasti kyberneettisiin ohjausjärjestelmiin, joihin kuuluvat erityisesti formaalit ja laskentapohjaiset suunnittelu ,  mittaus-, arviointi- ja palautejärjestelmät (Malmi & Brown 2008). Tutkimuksen painottuminen tällä tavoin johtunee osin siitä, että formaaleja ohjausjärjestelmiä on helpompi tutkia selkeinä ja objektiivisina kokonaisuuksina (Langfield-Smith 1997). Esimerkiksi budjetointia on tutkittu paljon erillisenä systeeminä. Tutkimuksessa voitaisiinkin ottaa yhä enemmän huomioon johdon ohjausjärjestelmät kokonaisuutena, jotta saataisiin selvitettyä, kuinka eri menetelmät täydentävät toisiaan. (Sandalgaard 2012) Myös Sandelin (2008) painottaa tutkimuksessaan johdon laskentajärjestelmien ja muiden ohjausmekanismien välisten yhteyksien tutkimista, jotta ohjauskokonaisuuden toimivuutta voitaisiin paremmin ymmärtää. 

Johdon ohjausjärjestelmiä on pyritty luokittelemaan myös niiden käyttömekanismien perusteella. Simonsin (1995) ohjausjärjestelmien luokittelun viitekehys (levers of control) kuvaa organisaation strategian toteutumiseen vaikuttavia tekijöitä, ja on yksi käytetyimmistä alan kirjallisuudessa (kuva 4). Alun perin se jakaantui neljään eri ohjausjärjestelmään: 1) uskomusjärjestelmät, 2) rajajärjestelmät, 3) diagnostiset sekä 4) interaktiiviset ohjausjärjestelmät. Päivitetyssä viitekehyksessä (Simons 2005) on näistä mukana kuitenkin vain kaksi: diagnostiset ohjausjärjestelmät sekä interaktiiviset ohjausjärjestelmät. Simonsin viitekehys käsittelee tapoja, joilla johto käyttää eri ohjausjärjestelmiä eri ohjausjärjestelmien erittelyn sijasta. Esimerkiksi erotus diagnostisten ja interaktiivisten systeemien välillä riippuu tavasta, jolla ylempi johto käyttää niitä. Toisin sanoen, Simonsin mukaan samaa ohjausjärjestelmää voidaan käyttää joko diagnostisesti tai interaktiivisesti.

\emph{Diagnostiset ohjausjärjestelmät} ovat muodollisia palautejärjestelmiä tulosten tarkkailuun. Niiden tavoitteena on suunnitellun strategian toteuttaminen, suorituskyvyn poikkeamien valvonta ja ohjaustoimenpiteiden päivittäminen. Keskeisessä osassa ovat tavallisesti kriittiset suorituskykymittarit (Simons 1995). \emph{Interaktiiviset ohjausjärjestelmät} keskittävät organisaation huomion strategian epävarmuuksiin, minkä lisäksi niillä voidaan hienosäätää ja muokata strategiaa markkinoiden muuttuessa. Ylin johto käyttää näitä keinoja säännöllisesti ja osallistuu tällä tavalla alaistensa päätöksentekoprosessiin. Interaktiiviset ohjausjärjestelmät synnyttävät vuorovaikutteisuutta johdon perustaessa uusia ohjelmia ja tavoitteita, tarkastellessa kuukausittain tulosta ja toimintasuunnitelmia ja seuratessa markkinoiden tilaa. (Simons 1995)

\section{Johtamisprosessi}

Johtamisprosessi pohjautuu johtamisjärjestelmään koostuen koko organisaation laajuudella toisiinsa kytkeytyvistä tapahtumista, joita yhteiset johtamisen työkalut ohjaavat (Kauppinen 2006). Useissa yrityksissä johtamisprosessi noudattelee yrityksen johtamisjärjestelmään perustuvan vuosikellon tapahtumia (Viitala & Jylhä 2007). Klassinen johtamisprosessi voidaan pelkistää seuraaviin tapahtumiin: suunnittelu, toimeenpano ja valvonta. Suunnitteluun kuuluvat esimerkiksi budjetit, investointilaskelmat, tuloskortit ja kilpailija-analyysit. Toimenpanoon ja valvontaan kuuluvat muun muassa tuotekustannuslaskelmat ja tarkkailua varten eroanalyysit, kustannusraportit sekä kuukausiraportit. (Järvenpää et al. 2010) Toisaalta johtamisprosessi voidaan nähdä johtamiskäytäntöjen ja -vastuiden verkostona, josta selviää käytäntöjen keskinäiset riippuvuudet ja vaikutussuhteet sekä niistä vastaavat toimijat. Johtamisprosessia toteutetaan koko organisaation laajuudella asetettujen tavoitteiden toteuttamiseksi. (Curtis et al. 2002; ref. Oiva 2007) Kun suunnitteluprosessia uudistetaan, on tärkeää ottaa huomioon vaikutukset yrityksen johtamisprosessiin. Organisaation johtamisprosessia on päivitettävä siten, että se tukee uuden suunnitteluprosessin käyttöä.

\section{Ennustamisesta}

Ennustamisen tarkoituksena on luoda mahdollisimman realistinen kuva tulevaisuuteen ja niin pitkälle kuin yrityksen johto suinkin pystyy näkemään (Lamoreaux 2011). Clarken (2007) mukaan ennustaminen kuuluu yritysten ydintoimintoihin, ja siihen tulisi osoittaa resursseja sen mukaisesti. Bourmistrovin ja Østergrenin (2011) mukaan ennustamisen tavoitteena on tunnistaa ristiriidat tavoitteen, toimintasuunnitelman ja todellisen tilanteen välillä ja saavuttaa riittävä tietämys suunnitelmien päivittämiseksi ja edelleen tavoitteiden saavuttamiseksi. Ennustaminen on tärkeää, sillä yritykset eivät pysty reagoimaan välittömästi muuttuviin olosuhteisiin tai tapahtumiin. Toisaalta näin ollen kyky reagoida nopeasti on merkitsevää, ei välttämättä kyky ennustaa virheettömästi. Ennustukset osuvat harvoin oikeaan, joten usein on viisaampaa ennustaa joukko useampia todennäköisiä tuloksia. Tällöin ennustuksia voidaan hyödyntää myös riskien arvioimisessa. (Barrett & Hope 2006)

Ennustamista tehdään jokaisessa yrityksessä. Usein ennustaminen on kuitenkin kiirehditty ja pintapuolinen tapahtuma (Fanning 1999). Epävirallisia ennusteita laaditaan usein kuukausittain tai neljännesvuosittain, mutta ne näyttävät tavallisesti vain päivitetyn näkymän loppuvuoden tulokseen. Toisin sanoen ennusteilla on harvoin yhteyttä yrityksen budjettiin liittyvään päätöksentekoon. (Akten et al. 2009) Tarve reaaliaikaisesta informaatiosta luo painetta säännöllisemmälle ennustamiselle. Viime vuosina teknologiat ovat kehittyneet myös talouden ennustamisessa ja suunnittelussa, minkä seurauksena yritykset ovat alkaneet vähitellen uudistamaan ennustamis- ja suunnitteluprosessejaan.

\section{Budjetointi}

Budjetointi on pohjimmiltaan tulevaisuuden suunnittelua. Sille on kuitenkin ajan kuluessa muodostunut useita eri käyttötarkoituksia. Tämän seurauksena on törmätty myös uusiin ongelmiin, sillä budjetointi ei ole riittävän joustava menetelmä suhteessa sen moninaisiin käyttötarkoituksiin. Perinteisen budjetoinnin ongelmakohdat ovatkin lähtökohtaisesti syy, miksi uusia budjetoinnin korvaavia menetelmiä on lähdetty kehittämään.

Budjetoinnin perusideana on luoda riittävän yksityiskohtainen rahamääräinen suunnitelma sovitulle ajanjaksolle liiketoiminnan työkaluksi. Eaton (2005) määrittelee budjetin määrälliseksi suunnitelmaksi, joka koskee tiettyä ajanjaksoa ja sisältää muun muassa suunnitellun myyntivolyymin ja liikevaihdon, resurssien määrän, kulut ja kustannukset, rahavirrat, varat ja vieraan pääoman määrän. Poikkeuksetta eri määritelmät mainitsevat budjetin olevan suunnitelma resurssien, kuten liikevaihto ja muut rahavirrat, hankinnasta sekä niiden käytöstä tietyn ajanjakson aikana (Etim & Agara 2011). Budjetti on edelleen yksi käytetyimmistä talousohjauksen ja -suunnittelun välineistä (Bhimani et al. 2008). Budjetointi myös kattaa vaikutukseltaan koko organisaation, ja sen voidaan katsoa olevan yksi tärkeimmistä yritysten johtamisjärjestelmäksi (Åkerberg 2006). Kokonaisuudessaan talousohjaukseen kuuluvat myös muut menetelmät ja toimintatavat, jotka antavat yritysjohdolle mahdollisuuden suunnitella ja seurata yrityksen taloudellista tilannetta ja sen kehittymistä.

Käsitteenä, toimintatapana ja menetelmänä budjetointi syntyi aikana jolloin yritysten toimintaympäristöt olivat suhteellisen staattisia verrattuna nykypäivään (Åkerberg 2006). Budjetoinnilla on näin ollen pitkä historia. Budjettien käyttö alkoi jo 1920-luvulla kulujen kontrollointiin ja rahavirtojen seuraamiseen (Michael 2007). Teollisen sarjatuotannon alkuaikoina yritykset saattoivat myydä lähes kaiken mitä pystyivät tuottamaan, ja haasteena oli siten lähinnä tuotannon rajakustannusten hallinta (Caulkin 2003). Toisaalta isojen diversifioituvien teollisuusyritysten oli pystyttävä päättämään resurssien allokoinnista siten, että osakkeenomistajille pystyttiin takaamaan suurin mahdollinen tuotto. Tätä tarkoitusta silmällä pitäen alettiin ennustaa eri liiketoimintayksiköiden potentiaalista taloudellista suorituskykyä, jotta pääoma voitaisiin osoittaa ensisijaisesti parhaiten tuottavaan liiketoimintaan. (Fanning 2000) Vähitellen budjetista muodostui myös tavoiteasetannan välikappale ja suorituskyvyn seurannan väline (Goode & Malik 2011). Globalisaation ja informaatioteknologian kehityksen myötä yritykset joutuvat toimimaan yhä dynaamisemmissa ja haastavammissa ympäristöissä, mikä on lisännyt tyytymättömyyttä budjetointiin. Harva johtaja on kuitenkaan lähtenyt haastamaan budjetointia ja sen asemaa tavoiteasetannan ydinjärjestelmänä (Åkerberg 2006). Vähitellen budjetoinnin rinnalle on kuitenkin alkanut nousta vaihtoehtoisia menetelmiä, jotka äärimmilleen vietynä mahdollistavat jopa budjetoinnista luopumisen.

\subsection{Budjetoinnin tehtävät ja tavoitteet}

Vaikka budjetti laaditaan tavallisesti taloudelliseksi ennusteeksi, sitä hyödynnetään myös muihin tarkoituksiin. Eri yritykset käyttävät budjetointia eri tarkoituksiin, ja sen avulla pyritään ratkaisemaan eri yrityksessä eri haasteita (Hansen 2010). Kun budjetointia lähdetään kehittämään, tai siitä on tarkoitus luopua jopa kokonaan, onkin tärkeää tunnistaa budjetoinnin vallitsevat käyttötarkoitukset. Tällöin niille voidaan osoittaa korvaava toiminto myös uudessa prosessissa.

Budjetin ensisijainen tarkoitus on sen määritelmän mukaisesti toimia suunnitelmana. Tavallisesti se kattaa tiivistettynä seuraavalle kalenterivuodelle odotettavissa olevat menot ja tulot. Suunnittelun yhteydessä tapahtuu usein myös resurssien allokointia. Linkki operatiiviseen suunnitteluun on ilmeinen, ja yksi tärkeä käyttötarkoitus budjetille onkin toimia apuvälineenä operatiivisessa suunnittelussa (Hansen & Van der Stede 2004). Budjetti saattaa määrittää jopa yrityksen operatiivisen toiminnan suunnan (Apanaschik 2007). Samalla budjetti toimii apuna koordinoinnissa ja eri tehtävien jakamisessa ja tasapainottamisessa vastuuyksiköille (De With & Dijkman 2008).

Hyvin yleisesti budjettia käytetään kulukontrolliin (Bourne 2005) ja rahan käytön auktorisointiin (De With & Dijkman 2008). Yksi selitys budjetoinnin suosiolle kontrollin apuvälineenä on se, että se kokoaa organisaation eri alueet yhteen dokumenttiin kattavaksi suunnitelmaksi (Hansen et al. 2003).

Budjettia käytetään usein myös ennusteena siitä, mitä tullaan saavuttamaan. Samanaikaisesti sitä käytetään tavoiteasetantaan (Bourne 2005; Drury 2008), suorituskyvyn arviointiin (Hansen & Van der Stede 2004; de With & Dijkman 2008) ja edelleen palkitsemisen perusteena. Suorituskykyä arvioidaan tyypillisesti suhteessa budjetissa asetettujen tavoitteiden saavuttamiseen (Drury 2008).

Budjettia voidaan hyödyntää strategian muokkaamisessa ja hienosäädössä (Hansen & Van der Stede 2004; Bourne 2005; Drury 2008). Joissakin yrityksissä budjetti toimii strategisen suunnitelman jalostumana (De With & Dijkman 2008). Myös Rickardsin (2006) mukaan budjetin olemassaolon syy on toimia apuvälineenä yrityksen strategian toteuttamisessa, ei niinkään kontrolloinnissa tai suunnittelussa.

Hyväksytty budjetti on erittäin käyttökelpoinen väline määrällisen informaation kommunikointiin liittyen suunnitelmiin ja rajoitteisiin (De With & Dijkman 2008). Druryn (2008) mukaan ylin johto käyttää budjettia ideoiden ja odotusten viestimiseen. Nykyaikaisessa yrityksessä budjetin ainut merkitys saattaa olla oleellisen informaation kommunikointi osakkeenomistajille ja muille ulkoisille osapuolille (Ekholm & Wallin 2010).

\subsection{Perinteisen budjetoinnin ongelmat}

Tämän päivän liiketoimintaympäristö on monilla toimialoilla huomattavan epävakaa. Osittain tämän seurauksena budjetointi menetelmänä on saanutkin osakseen huomattavan määrän kritiikkiä (Hope & Fraser 2000; Jensen 2001; Hansen et al. 2003; Lorain 2010). Perinteistä budjetointia on kuvailtu kalliiksi ja aikaa vieväksi prosessiksi, joka ei tue nopeaa ja joustavaa reagointia toimintaympäristön muutoksissa (Rickards 2008).  Åkerberg (2006) toteaa, että budjetointi on aikaa vievä, kallis ja useimmiten vain vähän arvoa lisäävä prosessi. Neely et al. (2001) ovat esittäneet kattavan yhteenvedon perinteisen budjetoinnin heikkouksista alan kirjallisuuden pohjalta. Budjetit

\begin{enumerate}
% You can use this command to set the items in the list closer to each other
% (ITEM SEParation, the vertical space between the list items)
\setlength{\itemsep}{0pt}
\item ovat kalliita laatia, sillä ne vievät paljon johdon aikaa
\item rajoittavat joustavuutta ja ovat usein muutoksen esteenä
\item ovat harvoin linjassa strategian kanssa
\item eivät lisää arvoa suhteessa niiden laatimiseen käytettyyn aikaan
\item keskittyvät kulujen vähentämiseen, eivät arvon luontiin
\item vahvistavat vertikaalista kontrollia
\item eivät vastaa nykyaikaisten verkostorakenteisten organisaatioiden tarpeita
\item mahdollistavat ”pelaamisen” ja epäterveen käyttäytymisen
\item laaditaan ja päivitetään liian harvoin, yleensä vuosittain
\item laaditaan perustelemattomien olettamuksien ja arvausten pohjalta
\item saavat ihmiset tuntemaan itsensä aliarvostetuiksi.
\end{enumerate}

Perinteisen budjetoinnin ongelmia käsitellään laajasti alan kirjallisuudessa. Erityisesti Hope ja Fraser (2003a) ovat dokumentoineet ja analysoineet kattavasti perinteisen budjetoinnin heikkouksia (Libby & Lindsay 2009), mutta he eivät ole ainoita, jotka ovat esittäneet kritiikkiä perinteistä budjetointia kohtaan (katso esimerkiksi Ekholm & Wallin 2000; Jensen 2001). Seuraavaksi tarkastellaan lähemmin budjetointia kohtaan esitettyä kritiikkiä.

\subsubsection{Budjetointisykli ja budjetin päivittäminen}

Budjetteja laaditaan ja päivitetään melko harvoin, tyypillisesti vuosittain. Harva yritys päivittää budjettia ennen seuraavaa budjetointikierrosta. (Hope & Fraser 1999) Huolellisesti laaditut taloudelliset ennusteet muuttuvatkin helposti käyttökelvottomiksi, jos niiden perustalla toimivat olettamukset sattuvat muuttumaan. Näitä ovat esimerkiksi korkotasot, valuuttakurssit ja raaka-aineiden hinnat. Tämän tyyppiset ulkoiset muutokset ovat nykyään yleisiä, minkä takia pitkän tähtäimen suunnittelu on riskialtista ja haastavaa. (Fanning 2000) Yleensä budjetointisykliä ei myöskään ole kytketty ”liiketoiminnan rytmiin”. Esimerkiksi pitkä suunnittelusykli nopeasti muuttuvalla toimialalla tai lyhyt suunnittelusykli suhteellisen vakaalla toimialalla ei palvele liiketoiminnan tarpeita (Fanning 2000).

Barret ja Hope (2006) tutkivat suunnittelukäytäntöjä 200 yrityksessä Iso-Britannian tuhannen suurimman yrityksen joukosta. Tutkimuksen mukaan huomattavan suuressa osassa yrityksiä (91 %) vuosibudjetti ei pätenyt enää loppuvuodesta. Lähes puolet raportoi tämän johtuvan ulkoisista tekijöistä, kuten markkinoiden käyttäytymisestä, kilpailusta ja talouden tilasta yleisesti. Vain 20 % arvioi budjettien olleen epätarkkoja tai liian optimistisia alun alkaen. Budjetin jäykkyys ja päivittymisen puute on näin ollen ongelma useissa organisaatioissa. Kun budjettitavoitteista jäädään, jyvitetään puuttuva erotus tyypillisesti jäljellä oleville vuosineljänneksille, aivan kuin koko vuoden näkymä säilyisi muuttumattomana (Myers 2001). Tällöin on kuitenkin hyvin epätodennäköistä, että alun perin suunniteltu tulos saavutettaisiin.

\subsubsection{Budjetoinnin mekanismit vanhentuneita}

Apanaschikin (2007) mukaan jopa yli puolet budjetointiin ja ennustamiseen kuluvasta työajasta kuluu vähemmän lisäarvoa tuottaviin tehtäviin, kuten tiedon keräämiseen ja yhdistelyyn, tiedon tarkastamiseen, hyväksyntään ja raporttien valmisteluun. Budjetointi koetaankin joissakin yrityksissä byrokraattiseksi lomakkeiden täyttelyksi, luovan lisäarvoa tuottavan ajattelun sijaan (Neely et al. 2001). Budjetointi lisää myös vain vähän arvoa suhteessa siihen käytettyyn aikaan (Michael 2007). Eräiden lähteiden mukaan jopa noin 20 % johdon ajasta kuluu suunnitteluun ja budjetointiin, mikä suuntaa ihmisten ajankäyttöä asioihin, joista saatavat hyödyt yritykselle ovat kyseenalaisia (Daum 2002).

Budjetointi vie aikaa monista eri syistä. Budjetteja käydään läpi useissa eri portaissa ja ne kulkevat edestakaisin ylemmän johdon ja keskijohdon välillä. Myös taulukkolaskennan hyödyntäminen ja tiedostojen siirtely paikasta toiseen vie turhaan aikaa. (Steed & Gu 2007) Lisäksi budjetteja laatiessa joudutaan etsimään tietoa pirstaloituneista tietolähteistä ja taulukoista (Fanning 2000), minkä lisäksi budjettien yhteenveto on työlästä (Michael 2007). Kun prosessi on muutenkin vaivalloinen, saa kulu-, tuote- ja strateginen kontrolli usein vain vähän huomiota prosessin aikana (Rickards 2008).

Budjetoinnin yhteydessä niiden pohjalla toimivia olettamuksia perustellaan usein rajallisesti. Ihmiset eivät välttämättä ajattele riittävän perinpohjaisesti ja eksplisiittisesti perustellen, kuinka he tulevat saavuttamaan tietyt luvut. (Neely et al. 2001) Jos budjetoitujen lukujen taustaa ja niiden muodostumisen perusteita ei tunneta tarkasti, on vaikea alkaa selvittää, mikä tulokseen on havaittujen poikkeumien kohdalla vaikuttanut (Åkerberg 2006).

\subsubsection{Vahvistetun budjetin vääristynyt merkitys organisaatioissa}

Budjettia käytetään usein tavoiteasetannan välineenä ja odotetun suorituksen mittarina (Steed & Gu 2007). Yksi tärkeimmistä kritiikin aiheista perinteistä budjetointia kohtaan onkin, että se toimii monissa yrityksissä vuoden pituisena sopimuksena odotetusta taloudellisesta suorituksesta (Prendergast 2000). Vuoden mittainen sitoumus on kuitenkin riski, mikäli se perustuu epävarmoihin ennusteisiin (Ekholm & Wallin 2000). Kun budjetti on kerran hyväksytty, niin tavoitteita voi olla vaikea muuttaa jälkeenpäin. Lorain (2010) huomauttaa, että yli puolet tutkituista yrityksistä ei pystynyt tekemään muutoksia budjetin hyväksymisen jälkeen. Lukkoon lyöty budjetti heikentää yrityksen ketteryyttä eli kykyä vastata liiketoimintaympäristön muutoksiin. Esimerkiksi ylimääräisten resurssien saaminen äkillisissä markkinatilanteen muutoksissa ei välttämättä ole mahdollista, mikäli resursseja ei ole ennalta budjetoitu. (Hope & Fraser 2003a) Vastaavasti jos tilanne muuttuu siten, että resursseja ei tarvittaisikaan niin paljon kuin on myönnetty, saatetaan ylimääräiset resurssit käyttää vain siksi, että niitä ei vähennettäisi seuraavana vuonna (McGee 2003; Lamoreaux 2011).

Jos yrityksen johtamiskulttuuriin kuuluu budjetin pitäminen suoritussopimuksena, ja tämän sopimuksen rikkominen nähdään vakavana asiana, on odotettavissa että ainakin osa budjettivastuullisista laatii tavoitteensa niin alhaisiksi, kuin mahdollista (Åkerberg 2006). Kun johtajien budjetoimat luvut vedetään yhteen, yrityksen laajuiset numerot saattavat olla kaukana todellisuudesta (Akten et al. 2009). Tällöin ylimmän johdon voi olla haastavaa saada realistista tietoa yrityksen suunnasta. Myös osakkeenomistajat arvostavat nimenomaan realistista informaatiota. Tavoiteasetannan näkökulmasta budjetin tulisi olla haastava, mutta luotettavan ennusteen saamiseksi budjetti olisi syytä laatia mieluummin varovaisuutta noudattaen. (Jukka Pellinen 2005) Tästä syntyy ristiriitoja realististen ja haastavien tavoitteiden asettamisen välille (Dugdale & Lyne 2004). Myös yrityksen kasvupotentiaalia voi jäädä hyödyntämättä, jos tavoitteet neuvotellaan varmuuden vuoksi hieman alakanttiin (Lamoreaux 2011). Vaikka tavoitteet asetettaisiinkin realistisesti, on nopeasti muuttuvassa taloudellisessa ympäristössä vaikeaa arvioida suoritusta reilusti, kun tulokseen saattaa vaikuttaa tapahtumat, joita ei pystytä ennakoimaan (Lorain 2010).

\subsubsection{Budjetointi kohdistaa huomion ensisijaisesti taloudellisiin seikkoihin}

Budjetointi keskittyy kulujen vähentämiseen arvon luonnin sijasta (Michael 2007), ja kiinnittää huomion lähinnä lukuihin, vaikka todelliset kilpailutekijät ovat usein aineettomissa tekijöissä (Åkerberg 2006). Voidaankin sanoa, että budjetointi keskittyy lähinnä taloudelliseen tulokseen muiden suoritusmittareiden kustannuksella. Michaelin (2007) mukaan budjetointi kytketään harvoin strategiaan ja on usein ristiriidassa siihen nähden. Budjetti saatetaan toisinaan laatia jopa huomioimatta yrityksen strategiaa ja tavoitteita (Fanning 2000).

Neely et al. (2001) toteavat, että budjetit keskittyvät tyypillisesti yrityksen sisäisiin asioihin, jolloin asiakkaiden palveleminen ja lisäarvon luominen eivät saa riittävästi huomiota. Budjetti keskittyy kuluvan vuoden tulokseen, kun huomion pitäisi yhtä lailla suuntautua liiketoiminnan kannattavuuteen pidemmällä aikavälillä. (Neely et al. 2001) Budjetit eivät varoitakaan tulevista ongelmista etukäteen (Åkerberg 2006), joten tulevaisuudennäkymien tarkasteluun tarvitaan erillinen työkalu. Osa kritiikistä strategiaan kytkeytymättömyydestä saattaa johtua budjetointimenetelmien kehittymättömyydestä. Jos budjetti laaditaan lisäämällä viime vuoden tulokseen sopiva prosenttikasvu, tai budjettia leikataan koko yrityksen tasolla, ei strategian painotuksia oteta asianmukaisesti huomioon (Hansen et al. 2003).

\subsubsection{Budjetointi suhteessa nykyaikaiseen organisaatiorakenteeseen ja -kulttuuriin}

Budjetoinnilla on historiallinen, rajoittava orientaatio (Dugdale & Lyne 2004). Sitä on perinteisesti käytetty keskitetyn johdon kontrollimekanismina, minkä takia budjetoinnin nähdään edelleenkin suosivan vertikaalista komentorakennetta (Hansen et al. 2003). Budjetointiprosessin avulla voidaan varmistaa, että johdon haluama olotila säilyy, ennemminkin kontrolloiden työntekijöitä kuin rohkaisten heitä (Neely et al. 2001).

Budjetit eivät vastaa nykyisten verkostorakenteisten organisaatioiden tarpeita. Yritykset hajauttavat toimintojansa ja hyödyntävät alliansseja ja kumppanuuksia yhä useammin tuottaakseen lisäarvoa ja palvellakseen asiakkaita parhaalla mahdollisella tavalla. (Neely et al. 2001) Budjetointi ei sovi myöskään matalan hierarkian organisaatioihin tai arvoketjuun perustuviin organisaatioihin, sillä se monimutkaistaa tarvittavaa päätöksentekoa (Hope & Fraser 2003a). Budjetit voimistavat osastojen rajoja tiedon jakamisen rohkaisun sijasta. Kun ihmiset pyrkivät omiin tavoitteisiinsa, jää aikaa vähemmän yhteistyölle ja synergioiden etsimiselle koko organisaation tasolla. (Neely et al. 2001)

Budjetit saavat ihmiset tuntemaan itsensä aliarvostetuiksi, kun heitä käsitellään budjeteissa minimoitavina kuluina, ennemmin kuin voimavaroina, joita kehitettäisiin eteenpäin (Hope & Fraser 1999). Jos lukuja asetetaan ylhäältä päin, ei suunnitelma ole liiketoiminnan itsensä laatima, mikä saattaa heikentää motivaatiota (Nolan 1999). Malli, joka keskittää huomion lisäarvon maksimoitiin kulujen minimoinnin sijasta, on paremmin linjassa informaatioajan johtamisfilosofioiden kanssa, kannustaa vastuullisuuteen sekä lisää luottamusta ja lojaaliutta (Brown & Atkinson 2001).

\subsection{Budjetoinnin yleisyys}

Dugdalen ja Lynen (2006) tutkimuksessa budjetti nähtiin keskeisenä elementtinä suunnittelu- ja kontrollimekanismien joukossa. Tutkimus käsitti 40 isobritannialaista yritystä. Tutkimuksen mukaan useimmat organisaatiot käyttävät budjetointia yhtenä organisaation kontrollimekanismina. (Dugdale & Lyne 2006) Voidaankin sanoa, että budjetointi on juurtunen syvälle organisaatioiden rakenteisiin. Budjetointi on keskitetysti koordinoitu aktiviteetti ja usein ainoa prosessi, joka kattaa organisaation kaikki toiminnot (Neely et al. 2001). Se nähdään edelleen organisationaalisena imperatiivina, ja budjetointikäytäntöjen muuttumisesta tai budjetoinnin hylkäämisestä on vain vähän empiiristä tutkimustietoa (De Waal et al. 2011).

Ekholm ja Wallin (2000) selvittivät suomalaisia yrityksiä koskevassa tutkimuksessaan, että suhteellisen harvat yritykset ovat hylänneet tai ovat aikeissa hylätä vuotuisen budjetoinnin kokonaan (14,3 %). Toisaalta huomattava määrä näin vastanneista yrityksistä yhtyivät keskeisiltä osin perinteistä budjetointia koskevaan kritiikkiin. Useat vastaajat myös kommentoivat, että vaihtoehtoisia tai täydentäviä systeemejä, kuten rullaava ennustaminen ja tuloskortit, on jo käytössä. (Ekholm & Wallin 2000)

Suuri osa Pohjoisamerikkalaisista yrityksistä näkee budjetoinnin edelleen hyödylliseksi ja budjetointia jatketaan siihen liittyvistä ongelmista huolimatta. Libby ja Lindsay (2009) toteavat tutkimuksessaan, että budjetoinnilla on edelleen olennainen merkitys yritysten yhtenä tärkeänä ohjausjärjestelmänä ja suurimmalla osalla yrityksistä ei ole suunnitelmia sen hylkäämiseksi. Monet yritykset suunnittelevat kuitenkin budjetointiprosessin kehittämistä havaittujen ongelmien korjaamiseksi. Libbyn ja Lindsayn (2007) aiemman tutkimuksen mukaan yli puolessa yrityksistä ylempi johto on sitä mieltä, että liiketoiminta ei selviäisi ilman budjetteja, ja ne ovat menestyksen edellytyksenä. Heidän mukaansa budjetointi on arvokasta yrityksille siihen kuluvasta ajasta ja rahasta huolimatta. Myös eurooppalaisista yrityksistä suurin osa käyttää budjetointia. Vuonna 1997 suurin osa eurooppalaisista yrityksistä ilmoitti käyttävänsä formaalia budjetointiprosessia (Hope & Fraser 1997). Ruotsissa Glader et al. (1996) selvittivät, että 89 % yrityksistä käyttää budjetointia. Kuitenkin 40 % ilmoitti, että muutoksia budjetointiprosessiin on tulossa, suurimpana rullaavien ennusteiden hyödyntäminen. (Glader et al. 1996; ref. Ekholm & Wallin 2000) Toisaalta etenkin Pohjoismaissa on paljon yrityksiä, jotka ovat hylänneet kokonaan budjetoinnin. Näitä ovat esimerkiksi Svenska Handelsbanken, Volvo, Rhodia ja Borealis. (Hope & Fraser 2000)

\chapter{Vaihtoehtoja perinteiselle budjetoinnille}
\label{chapter:vaihtoehtoja}

Informaatioteknologian kehityksestä ja globalisaatiosta johtuen, ovat nykyajan liiketoimintaympäristöt huomattavasti nopealiikkeisempiä kuin aikaisemmin. Samalla maailmanlaajuinen taloudellinen tilanne on epävarmempi, ja kysyntä saattaa heitellä totuttua nopeammin. Ne organisaatiot, jotka jatkavat perinteistä budjetointia käyttäen siihen kuukausia, tulevat jäämään kilpailijoidensa taakse. Organisaatiot saattavat kuitenkin olla haluttomia hylkäämään kokonaan vuosittaista budjetointia. Yksityiskohtaista budjettia käytetään tavallisesti myös ulkoiseen raportointiin, ja uusien menetelmien käyttöönotto ei poista tätä tarvetta (Hunt 2003). Toisaalta De Waalin (2011) mukaan organisaatiossa on vallittava tietty tyytymättömyyden taso, ennen kuin suunnitteluprosessin nykytilaa aletaan selvittää, ja vaihtoehtoja tutkia. Kirjallisuuden perusteella voidaan joka tapauksessa sanoa, että perinteiseen budjetointiin kohdistuvan kritiikin määrän ja uusien menetelmien käyttöönottaneiden organisaatioiden määrän välillä ei näytä olevan selvää yhteyttä.

Cranfieldin yliopiston ja Accenturen yhdessä vuonna 2001 tekemän tutkimuksen mukaan yritykset ovat ottaneet hyvin erilaisia lähestymistapoja suunnittelu- ja budjetointiprosessiensa kehittämiseen. Tutkimukseen osallistui 15 yhdysvaltalaista ja eurooppalaista yritystä, joissa oli jo aloitettu suunnittelu- ja budjetointikäytäntöjen kehittäminen. Tutkimuksessa ei löytynyt yhtä tapaa kehitystyölle, vaan yritykset toimivat eri lähestymistavoilla ja prioriteeteilla, valiten omiin tarpeisiinsa soveltuvia menetelmiä. (Bourne 2004) Yritysten onkin punnittava tarkasti eri vaihtoehtoja ja menetelmiä, joista osa perustuu budjetointiprosessin maltilliseen kehittämiseen, osan suositellessa budjetoinnin hylkäämistä kokonaan ja siirtymistä vaihtoehtoisten menetelmien käyttöön. Riippuen lähtötasosta, joillekin organisaatioille riittää nykyisten toimintojen tehostaminen, kun toisten on mahdollisesti alusta alkaen mietittävä suunnittelun ja ennustamisen strategista painoarvoa, sekä näihin liittyviä prosesseja, tietomalleja ja työkaluja (Apanaschik 2007).

Kun suuri osa yrityksistä on vasta vähitellen alkanut pohtimaan budjetointijärjestelmän kehittämistä tai sen vaihtoehtoja, ovat edistyneemmät yritykset jo liikkeellä. Leahyn (2006) mukaan kuilu budjetointi¬-, suunnittelu- ja ennustamiskyvykkyydessä johtavien yritysten ja muiden yritysten välillä jatkaa kasvamistaan. Yritysten on siis syytä miettiä suunnitteluprosessiensa tehokkuutta.  Huippuluokan yritykset hyödyntävät parhaita käytäntöjä (best practices), kuten esimerkiksi rullaavaa ennustamista, yrityksen laajuista toiminnanohjausjärjestelmää, toimintopohjaista budjetointia ja analyyttisiä sovelluksia. (Leahy 2006) Kehitysprojektien yleisiä tavoitteita ovat muun muassa

\begin{itemize}
% You can use this command to set the items in the list closer to each other
% (ITEM SEParation, the vertical space between the list items)
\setlength{\itemsep}{0pt}
\item päätöksenteon parantaminen
\item ennustettavuuden parantaminen
\item tehokkaan johtamistyökalun luominen
\item arvokkaan ja helposti hyödynnettävän näkemyksen parempi saatavuus
\item suunnittelusyklin lyhentäminen
\item informaation läpinäkyvyyden ja kompetenssin kasvattaminen
\item operatiivisten tulosten parantaminen
\item resurssitarpeiden vähentäminen ja
\item paremman tuen saaminen talousosastolta (Moriarty 2001).
\end{itemize}

Libby ja Lindsay (2009) argumentoivat, että budjetoinnin kehittämiseen ei tulisi ottaa joko–tai-näkökulmaa. On olemassa esimerkkejä hyvin menestyksekkäistä eri menetelmiä hyödyntävistä. Saattaakin olla hedelmällisempää kehittää kunkin mallin mahdollisuuksia hakemalla parempaa ymmärrystä eri menetelmien taustalla toimivista mekanismeista ja prosesseista menestyksekkäissä yrityksissä. Fanningin (2000) mukaan kaksi viimeaikaista trendiä budjetoinnin kehittämisessä ovat olleet budjetointitiheyden kasvattaminen, käytännössä rullaava budjetointi tai suunnittelu, sekä budjetoinnin merkityksen kokonaan kyseenalaistava beyond budgeting. Kolmas, jo jalansijaa saanut menetelmä on toimintopohjainen budjetointi. Muita suunnittelun tukena käytettyjä menetelmiä ovat skenaariosuunnittelu, nollapohjabudjetointi sekä arvoperusteinen johtaminen.

\section{Rullaava suunnittelu}

Rullaavien ennustamis- ja suunnittelukonseptien yhteydessä on syytä varmistaa, että yrityksen sisällä vallitsee yhtenäinen näkemys eri termien tarkoituksista. Tämä ei ole aina itsestään selvää. (Åkerberg 2006) Rullaavan ennustamisen, rullaavan suunnittelun ja rullaavan budjetoinnin välistä eroa onkin syytä tarkastella lyhyesti. Perusidea niissä on sama – ennustus- tai suunnitteluhorisontin jatkaminen sen hetkisen tilikauden jälkeiselle ajalle.
Rullaavalla ennustamisella (rolling forecasting, continuous forecasting) viitataan ennusteiden säännölliseen päivittämiseen, tyypillisesti kuukauden tai vuosineljänneksen välein. Tämä voi tapahtua esimerkiksi päivittämällä ennusteet kuukausittain toteumatiedon pohjalta. Samalla ennustehorisonttia jatketaan aina siten, että siihen sisältyvien ennustejaksojen määrä säilyy vakiona. Kun uusin ennustekausi päivittyy,  tarkentuu samalla koko ennustehorisontti myös jaksoa edeltävien lukujen osalta (Clarke 2007) Olennaista rullaavassa ennustamisessa on se, että ennustehorisontti on kuluvaa tilikautta pidemmällä (Åkerberg 2006). Rullaava ennustaminen on tavallisesti rajoittunut tarkistettujen taloudellisten arvioiden tuottamiseen (Fanning 2000).
Lorainin (2010) mukaan pelkkä rullaava ennustaminen ei täysin korvaa budjetointia, sillä se ei ulotu sellaisenaan suorituksen arvioinnin ja motivoinnin tarkoituksiin. Rullaava ennustaminen voidaan kuitenkin kytkeä saumattomasti rullaavaan suunnitteluun.

Rullaava suunnittelu pitää taloudellisten ennusteiden lisäksi sisällään operatiivisten suunnitelmien päivittämisen sekä tarpeen tullen strategisten vaihtoehtojen ja suunnitelmien tarkistamisen. Rullaavaa ennustamista tehdään usein rullaavan suunnittelun yhteydessä, jolloin ne tukevat luonnollisesti toisiaan. (Fanning 2000) Kuten rullaavaa ennustamista, myös rullaavaa suunnittelua tehdään säännöllisin väliajoin kalenterivuoden aikana. Aina, kun uutta toteumatietoa ja ennustetta saadaan käytettäväksi, jatketaan suunnitteluhorisonttia siten, että suunnittelujaksojen määrä säilyy vakiona. Säännöllisen ennustamisen pohjalta suunnitelmien päivittäminen helpottuu. Kun tavoitteelliset suunnitelmat asetetaan useamman kerran kalenterivuoden aikana, yrityksen reagointikyky kasvaa ja ympäristön muutoksiin pystytään vastaamaan aikaisempaa tehokkaammin. Mikäli rullaavaa suunnittelua käytetään koko organisaatiossa ja se kytketään onnistuneesti johtamisjärjestelmään, ei budjetoinnille välttämättä ole enää tarvetta.

Rullaavan suunnittelun voidaan kirjallisuuden perusteella katsoa olevan rullaavan budjetoinnin synonyymi. Tässä tutkimuksessa ja S ryhmässä on termiksi valikoitunut kuitenkin rullaava suunnittelu. Budjetointi siinä mielessä, kun se perinteisesti käsitetään, ei sovellu tehtäväksi useampia kertoja vuodessa. Näin ollen on selvempää käyttää termiä ”rullaava suunnittelu” viitattaessa nykyaikaiseen useampia kertoja vuodessa tapahtuvaan suunnitteluun, jonka tarkoituksena on korvata budjetointi. Lisäksi jotkut katsovat, että budjetointiin terminä liittyy siinä määrin kielteisiä mielleyhtymiä, etteivät suosittele termin käytön jatkamista suunnitteluprosessin kehittämisprojektien yhteydessä.

Rullaavuuden idea ei ole uusi, sillä monissa yrityksissä on jo pitkään tehty budjettikauden sisäistä ennustamista tavallisen vuosibudjetoinnin lisäksi (Ekholm & Wallin 2000). Nämä ennusteet ulottuvat kuitenkin tavallisesti vain kalenterivuoden loppuun. Perinteisesti yritykset ovatkin sitoneet budjetointinsa tilikauteen. Rullaavaan suunnittelun tärkein oivallus on poistaa tämä kytkös. Näin ollen myös pitkäaikaisia suunnitelmia voidaan muuttaa vuoden mittaan ilman keinotekoisia rajoja, kuten tilikauden päättyminen (Apanaschik 2007). Kuvassa 5 havainnollistetaan perinteisen budjetoinnin ja rullaavan suunnittelun suunnitteluhorisontteja.

Åkerbergin (2006) mukaan rullaavuudella tavoitellaan tavallisesti muun muassa vuosituloksen parempaa ennustamista, operatiivisen ohjauksen terävöittämistä, aikaisempaa reagointia, siirtymistä pois kalenteriohjauksesta, suunnittelun tihentymistä, kassavirran parempaa hallintaa ja tarpeettoman työn vähentämistä. Lorainin (2010) mukaan tavallisimmat syyt rullaavan suunnittelun käyttöönottamiselle ovat

\begin{itemize}
% You can use this command to set the items in the list closer to each other
% (ITEM SEParation, the vertical space between the list items)
\setlength{\itemsep}{0pt}
\item budjetoinnin heikkoudet
\item talouden hallinnan parantaminen
\item operatiivisen johtamisen kehittäminen
\item päätöksentekoprosessin parantaminen sekä
\item työn kohdistaminen arvoa lisääviin toimintoihin.
\end{itemize}

Lisäksi Lorain (2010) selvitti tutkimuksessaan rullaavan suunnittelun tehtäviä. Vaikka rullaavaan suunnitteluun siirtyvät yritykset näkivät ympäristön muutoksiin sopeutumisen tärkeänä syynä uuden menetelmän käyttöönottoon, olivat päätehtävät haastatelluissa yrityksissä suunnittelu, talouden johtaminen ja hallinta, operatiivinen johtaminen ja oppiminen (taulukko 1).

\begin{table}
\begin{tabular}{|p{3cm}|p{10cm}|}
% Alignment of sells: l=left, c=center, r=right.
% If you want wrapping lines, use p{width} exact cell widths.
% If you want vertical lines between columns, write | above between the letters
% Horizontal lines are generated with the \hline command:
\hline % The line on top of the table
\textbf{Tehtävä} & \textbf{Tavoitteet} \\
\hline
% Place a & between the columns
% In the end of the line, use two backslashes \\ to break the line,
% then place a \hline to make a horizontal line below the row
Suunnittelu &
\begin{itemize}
\setlength{\itemsep}{0pt}
\item Operatiivisen toiminnan kytkeminen strategiaan
\item Asetettujen tavoitteiden saavuttaminen
\item Uusien tuotteiden ja palveluiden kehittäminen
\end{itemize}
\\
\hline

Talouden johtaminen &
\begin{itemize}
\setlength{\itemsep}{0pt}
\item Jatkuvat rahavirtapäivitykset
\item Kommunikointi osakkeenomistajille
\item Taloudellinen viestintä
\end{itemize}
\\
\hline

Operatiivinen johtaminen &
\begin{itemize}
\setlength{\itemsep}{0pt}
\item Resurssien allokointi tai jäädyttäminen
\item Operatiivinen suunnittelu
\item Toimitusketjun koordinointi
\item Kulukontrolli
\end{itemize}
\\
\hline

Oppiminen ja tietämys &
\begin{itemize}
\setlength{\itemsep}{0pt}
\item Parempi näkyvyys
\item Ympäristön ymmärtäminen
\item Nopeampi päätöksentekosykli
\item Tuloskeskeinen kulttuuri
\item Sisäinen viestintä ja keskustelu
\end{itemize}
\\
\hline
\end{tabular} % for really simple tables, you can just use tabular
% You can place the caption either below (like here) or above the table
\caption{Research methodology courses}
% Place the label just after the caption to make the link work
\label{table:päätehtävät}
\end{table} % table makes a floating object with a title

\section{Skenaariosuunnittelu}

Skenaariosuunnittelu (scenario planning) on menetelmä, jonka avulla voidaan kehittää konkreettisia makroekonomisia liiketoimintaskenaarioita, ja mallintaa niiden vaikutuksia omaan, asiakkaiden ja jopa kilpailijoiden toimintaan. Lisäksi voidaan tunnistaa erilaisia tapahtumia, kuten lyhytaikaisen rahoituksen saatavuudessa tapahtuvat muutokset tai tietyn markkinaosuuden pieneneminen, jotka laukaisevat vaihtoehtoisia skenaarioita. (Akten et al. 2009)

\section{Nollapohjabudjetointi}

Nollapohjabudjetointi (zero-based budgeting) kehitettiin alun perin 1960-luvun loppupuolella Yhdysvalloissa liittohallituksen käyttöön, ja se on edelleenkin käytössä etenkin julkisen sektorin puolella (Fanning 2000). Menetelmä perustuu budjetointiprosessin aloittamiseen puhtaalta pöydältä, ottaen huomioon aina sen hetkiset olettamukset kunkin liiketoiminnan kehitysnäkymistä. Tämän jälkeen operatiiviset kulut ja pääomakulut priorisoidaan yrityksen strategian ja oletetun pääoman tuoton mukaan. Budjetin paloitteleminen tarkkojen rahoituspäätöksien mukaiseksi helpottaa johtoa edelleen tekemään päätöksiä kilpailevien hankkeiden välillä resurssien rajallisuuden edessä. (Akten et al. 2009)

Menetelmän mukaan resurssien tarve on perusteltava uudelleen joka vuosi. Jos ympäristö on muuttunut, voidaan tarpeettomat resurssit eliminoida. Käytännössä monia resursseja tarvitaan toistuvasti, joten sama työ joudutaan tekemään vuosittain uudestaan. (Fanning 2000) Menetelmän ongelmana onkin, että kulujen perusteleminen ja niistä päättäminen jokaisen budjetointikierroksen yhteydessä vie paljon aikaa (McNally 2002).

\section{Arvoperusteinen johtaminen}

Arvoperusteinen johtaminen (value-based management) perustuu osakkeenomistajan taloudellisen lisäarvon laskemiseen budjetoinnin lähtökohtana. Näin pyritään ottamaan huomioon kannattavuus niin lyhyellä kuin pidemmälläkin aikavälillä (Goode & Malik 2011). Menetelmälle on tyypillistä muodollinen ja systemaattinen lähestyminen suunnitteluun, esimerkiksi eri kuluerät käsitellään projektialoitteina, joita arvioidaan sen perusteella, miten ne luovat taloudellista lisäarvoa osakkeenomistajille. (Neely et al. 2003)

\section{Toimintopohjainen budjetointi}

Toimintopohjainen budjetointi (activity-based budgeting) on toiminto- ja kapasiteettipohjaisten konseptien sovellus talouden suunnitteluun ja budjetointiin (Hansen 2010). Toimintopohjaisen budjetoinnin kehitys eriytyi toimintolaskennasta (activity-based costing) omaksi alueekseen 1980-luvun loppupuolella (Fanning 2000). Pääasiassa yhdysvaltalaisen The Consortium of Advanced Management Internationalin (CAM-I) kehittämä budjetointimenetelmä tähtää budjetoinnin parantamiseen keskittymällä operatiivisen suunnittelun ja taloudellisen tuloksen suhteeseen. Toimintopohjaisessa budjetoinnissa keskeistä on yksityiskohtainen tieto aktiviteeteistä. Suunnittelu ja kontrolli painottuvat kuluajureihin ja kehitystä tarvitseviin prosesseihin. Nämä ajurit ja prosessit identifioidaan toimintolaskennan avulla. (De With & Dijkman 2008) Tämä myös kytkee budjetoinnin tehokkaasti yrityksen operatiiviseen toimintaan. Koska menetelmä vaatii käyttää toimintolaskennan hyödyntämistä, on harvalla yrityksellä suoraan valmiuksia menetelmän hyödyntämiselle (Hansen 2010).

Toimintopohjaisen budjetoinnin etuna on, että se sitoo yrityksen toiminnan suunnittelun eksplisiittisesti yrityksen eri prosesseihin. Tavallisesti toimintopohjaisen budjetoinnin mallit toteutetaan näitä tukevien ohjelmistojen avulla, jolloin sykliä voidaan toistaa helposti ympäristön muutoksissa. (Hansen 2010) Esimerkiksi tasapainottamalla operatiiviset vaatimukset, vältetään tarpeettomien taloudellisten vaikutusten laskemisen operatiivisesti epäkelvoille suunnitelmille. Toimintopohjainen budjetointi keskittyy budjetin luomiseen eksplisiittisesti toimintojen ja resurssien pohjalta. Menetelmä auttaa näin operatiivisen tason johtajia ja työntekijöitä ymmärtämään helpommin mihin budjetti perustuu, jolloin he pystyvät keskustelemaan budjettiin liittyvästä informaatiosta operatiivisin termein taloudellisten termien sijasta. Vastaavasti tarjoamalla ymmärrystä resurssien ja toimintojen kytköksistä, toimintopohjainen budjetti auttaa johtajia ymmärtämään, kuinka he suoriutuvat työssään. (Hansen et al. 2003)

\section{Beyond budgeting}

Beyond budgeting -malli on herättänyt paljon keskustelua alan kirjallisuudessa. Sen lupauksena on nykypäivän vaatimuksiin vastaava organisaatio, jossa perinteistä budjetointia ei enää tarvita. Mallin kehittäjät Jeremy Hope ja Robin Fraser ovat vuosien ajan kritisoineet perinteistä budjetointia tehdessään työtään tunnetuksi (Goode & Malik 2011). Beyond budgeting ei ole ainoastaan työkalu taloudellisen suunnittelun ja raportoinnin tarpeisiin, vaan sen avulla voidaan johtaa liiketoimintaa joustavalla ja dynaamisella tavalla. Se onkin eräänlainen viitekehys uuden johtamisjärjestelmän rakentamiseksi. (Daum et al. 2005) Mitään yksittäistä työkalua malli ei tarjoa, vaan se perustuu joukkoon organisaation toimintaa ohjaavia periaatteita. Toisaalta näiden kautta ohjataan eri työkalujen käyttöön, joita ovat esimerkiksi rullaava ennustaminen, tuloskortit, vastaaviin yrityksiin vertailu (benchmarking) ja yrityksen laajuiset tietojärjestelmät (Bunce et al. 2002; ref. Becker et al. 2009)

Beyond budgeting -mallia kehitettiin aluksi CAM-I:n alaisuudessa, mutta Lontoossa vuonna 1998 kehitystä jatkamaan perustettiin The Beyond Budgeting Round Table (BBRT). Kehitystyö perustui pitkälti perinteisestä budjetoinnista vaihtoehtoisiin menetelmiin siirtyneiden yritysten tutkimiseen. Tämän pohjalta laadittiin 12 periaatteen lista (taulukko 2), joka muodostaa mallin perustan. Yhdessä ne muodostavat käytännön pohjalta rakentuvan viitekehyksen, jonka avulla on mahdollista selvittää organisaation nykytilaa, ja edelleen ohjata sitä kohti vaihtoehtoisia johtamismalleja. Ne myös muodostavat holistisen mallin, joka yhdistää kilpailulliset menestystekijät, johtajuusajattelun, johtamisprosessin ja tietojärjestelmät, ja jonka avulla johtajat voivat luoda yhtenäisen organisaation. (Hope et al. 2007)

\begin{table}
\begin{tabular}{|p{3cm}|p{10cm}|}
% Alignment of sells: l=left, c=center, r=right.
% If you want wrapping lines, use p{width} exact cell widths.
% If you want vertical lines between columns, write | above between the letters
% Horizontal lines are generated with the \hline command:

\hline  % The line on top of the table
\multicolumn{2}{|p{13cm}|}{\textbf{Johtamisperiaatteet} \emph{(leadership principles)}} \\
\hline
Asiakaslähtöisyys & Kannusta kaikkia asiakaslähtöisyyteen, ei hierarkkisiin suhteisiin \\
\hline
Organisaatio & Organisoi työntekijät vastuutettuihin tiimeihin, ei keskitettyjen toimintojen ympärille \\
\hline
Vastuullisuus & Kannusta ihmisiä johtajuuteen, ei pelkkään suunnitelman seuraamiseen \\
\hline
Autonomia & Anna tiimeille vapaus ja resurssit toimia, älä mikromanageroi \\
\hline
Arvot & Johda muutamien selkeiden arvojen, tavoitteiden sekä rajojen kautta, ei yksityiskohtaisten sääntöjen ja budjettien \\
\hline
Läpinäkyvyys & Kannusta avoimeen tiedonvälitykseen ja anna vastuuta, älä rajoita hierarkkisesti \\
\hline
\multicolumn{2}{|p{13cm}|}{\textbf{Prosessiperiaatteet} \emph{(process principles)}} \\
\hline
Tavoitteet & Aseta suhteelliset tavoitteet jatkuvaan suorituksen parantamiseen, älä neuvottele kiinteitä suoritustavoitteita \\
\hline
Palkitseminen & Palkitse jaettu menestys suhteellisten suorituskykymittareiden perusteella, ei kiinteiden tavoitteiden perusteella \\
\hline
Suunnittelu & Tee suunnittelusta jatkuva ja osallistava prosessi, ei vuosittainen ylhäältä–alas -tapahtuma \\
\hline
Kontrolli & Perusta kontrolli suhteellisiin mittareihin ja trendeihin, ei poikkeumiin vuotuisesta suunnitelmasta \\
\hline
Resurssit & Vapauta resursseja tarpeen mukaan, ei kerran vuodessa budjetein säädeltynä \\
\hline
Koordinaatio & Koordinoi eri vuorovaikutuksia dynaamisesti, ei vuosittaisen suunnittelusyklin mukaan \\
\hline

\end{tabular} % for really simple tables, you can just use tabular
% You can place the caption either below (like here) or above the table
\caption{Beyond budgeting -johtamisperiaatteet (Hope et al. 2007).}
% Place the label just after the caption to make the link work
\label{table:bb}
\end{table} % table makes a floating object with a title

Suhteelliset suoritusstandardit suorituskyvyn arvioinnissa kuuluvat olennaisesti malliin (de With & Dijkman 2008; Hansen et al. 2003). Menetelmän mukaisesti organisaation yksikköjä ja niiden johtajia arvioidaan vertaamalla suoritusta kilpailijoiden suoritukseen olennaisten mittareiden osalta (Hansen et al. 2003). Menetelmä kehottaa erityisesti luopumaan kiinteistä tai kiinnitetyistä tavoitteista, jolloin tarve numeroiden manipuloinnille jää pois (Daum et al. 2005). Myöskään uhka siitä, mitä tapahtuu, jos tavoitteisiin ei päästä, ei rajoita tällöin samalla tavalla johtajien halukkuutta innovatiivisiin ratkaisuihin tai riskinottoon sopivassa tilanteessa. (Hope & Fraser 2003a) Useimmilla yrityksillä ei kuitenkaan välttämättä ole riittävän hyvää dataa kilpailevien yritysten suorituksesta, johon omaa suoritusta voisi verrata. Tämä on tyypillistä etenkin tiukasti kilpailuilla aloilla, joissa suhteellinen suorituskyvyn mittaus olisi potentiaalisesti kaikista hyödyllisin. Vaikka teoreettisessa kirjallisuudessa on pohdittu, millaisiin tilanteisiin suhteellinen suorituskyvyn mittaaminen (relative performance evaluation) sopisi parhaiten, on empiirinen tutkimus aiheesta vielä puutteellista. Tämä on myös hidastanut suhteellisen suorituskykymittariston implementointia yrityksissä. (Hansen et al. 2003)

Menetelmän yhtenä tavoitteena on tehdä liiketoimintayksiköistä pienempiä ja kannustaa niitä yrittäjämäiseen toimintaan. Kun pienet tiimit saavat hyödyntää tehokkaasti avautuvia mahdollisuuksia liiketoiminnalle, pystyy organisaatio sopeutumaan nopeasti ympäristön muutoksiin. Samalla strategia muovautuu alhaalta asti tehokkaalla tavalla. (Hope & Fraser 2003b) Organisaation radikaali hajauttaminen ylittää kuitenkin taloushallintoa tutkivan kirjallisuuden perinteiset rajat. Kun puhutaan organisaation arkkitehtuurin ja johtamisjärjestelmän muokkaamisesta, johon kuuluu esimerkiksi vastuun delegointi yksilöille, palkitsemismekanismien muovaaminen sekä suorituskyvyn arvioimiseen tarkoitettujen systeemien päivittäminen, nousee yritysten kynnys menetelmän käyttöönottamiseksi. (Hansen et al. 2003) Koko johtamisjärjestelmän uudistaminen on huomattavan radikaali ja riskialtis hanke verrattuna yksittäisen työkalun tai tekniikan lisääminen johdon työkalupakkiin (Becker et al. 2009). Esimerkiksi tuloskorttia tai vastaavia työkaluja on mahdollista kokeilla muuttamatta oleellisesti yrityksen johtamiskäytäntöjä (Modell 2009).

\chapter{Perinteisestä budjetoinnista rullaavaan suunnitteluun}

Rullaavan suunnittelun käyttöönotto ei välttämättä vaadi mitään fundamentaalia muutosta verrattuna siihen, miten yritys on aikaisemmin valmistanut budjettinsa. Suurin ero liittyy suunnittelun muuttumiseen ympäri kalenterivuotta tapahtuvaksi jatkuvaksi suunnitteluksi. (Myers 2001) Kuitenkin, mikäli yritys on päättänyt siirtyä rullaavaan suunnitteluun, halutaan suunnitteluprosessia todennäköisesti kehittää muiltakin osin. Rullaavaan suunnitteluun siirtyminen on syytä suunnitella huolella. Projektin suunnittelu ja käyttöönotto kestää helposti useita vuosia lähtötilanteesta riippuen.  Tärkeintä on selvittää huolellisesti, mihin rullaavalla suunnittelulla organisaatiossa pyritään ja pitää nämä tavoitteet kirkkaana mielessä. (Åkerberg 2006)

Moriartyn (2001) mukaan onnistuneessa suunnitteluprosessin kehittämisprojektissa tulee kiinnittää huomio seuraaviin alueisiin:

\begin{itemize}
% You can use this command to set the items in the list closer to each other
% (ITEM SEParation, the vertical space between the list items)
\setlength{\itemsep}{0pt}
\item ennustaminen
\item tavoiteasetanta
\item operatiivinen suunnittelu
\item johdon raportointi
\item suorituskyvyn hallinta
\item strateginen suunnittelu
\item palkitseminen
\item talouden ja teknologian suhde ja tehtävät.
\end{itemize}

Lorainin (2010) tutkimuksen haastattelujen perusteella keskeiset onnistumisen edellytykset rullaavan suunnitteluprosessin käyttöönotolle ovat vastaavasti johdon sitoutuminen, tavoitteiden viestittäminen, kytkeminen strategiaan ja tietotekniikkaosaston tuki. Myös Bourne (2004) esittää yhteisiä tekijöitä liittyen onnistuneisiin suunnitteluprosessin uudistamis- tai kehittämishankkeisiin. Näitä olivat ulkoisten tekijöiden ja ympäristön parempi huomioiminen, keskittyminen strategiaan, yrityksen laajuiseen tietojärjestelmään investointi sekä erilliset mallit ennustamiseen.

Kuten edellisessä luvussa kävi ilmi, on erilaisia lähestymistapoja suunnitteluprosessin uudistamiseen useita erilaisia. Vaatimukset uuden suunnitteluprosessin implementoinnissa vaihtelevat organisaation ja sen kykyjen mukaan. Näin ollen esimerkiksi rullaava suunnittelu tai jatkuva ennustaminen saattaa osoittautua hyödylliseksi yhdessä yrityksessä, mutta ei seuraavassa (Rickards 2008). Tässä luvussa esitellään keskeisiä rullaavaan suunnitteluun liittyviä näkökulmia, ja kerrotaan, mikä niiden merkitys on, ja millä tavoin ne tulisi ottaa huomioon rullaavaan suunnitteluun siirtyessä.

\section{Rullaavan suunnittelun ominaispiirteet}

\subsection{Suunnitteluhorisontin pidentyminen}

Rullaavan suunnittelun avulla pyritään luomaan näkymä sekä lyhyelle että keskipitkälle aikavälille (Lorain 2010). Suunnitteluhorisontin pituuden valintaan vaikuttavat useat eri seikat. Kuuden vuosineljänneksen mittainen suunnitteluhorisontti on yleinen, mutta myös muita vaihtoehtoja pituuksia käytetään yrityksissä. Esimerkiksi S ryhmässä on päädytty 21 kuukauden eli seitsemän vuosineljänneksen suunnitteluhorisonttiin. Sopiva pituus riippuu yrityksen toimintaympäristöstä ja markkinoista, mutta kyky ja mahdollisuudet lisätä tai vähentää esimerkiksi tuotantokapasiteettia joustavasti voi olla merkitsevää suunnitteluhorisontin valinnalle (Fanning 2000). Suunnitteluhorisontin pituudelle ei ole yhtä oikeaa vaihtoehtoa. Sen tulisi kuitenkin olla pidempi kuin sen hetkinen tilikausi. Jossain määrin valintaan vaikuttaa myös operaatioita, kapasiteettia ja pääoman kulutusta koskeva päätöksentekonopeus. (Barrett & Hope 2006)

\subsection{Suunnittelutiheyden lisääminen}

Rullaavan suunnittelun yksi vahvuus on joustavan suunnittelun ja tavoiteasetannan mahdollistuminen, mikä edistää nopeaa reagointia ympäristön muutoksiin. Joustavuus perustuu pääasiassa siihen, että suunnittelutiheyden muutoksen ansiosta myös päätöksentekosykli lyhenee. Rullaava suunnittelu tekee organisaatiosta dynaamisemman ja auttaa johtajia käsittelemään uhkia ja mahdollisuuksia niiden ilmaantuessa. Rullaava suunnittelu on siten tehokas työkalu operatiiviselle johdolle. (Lorain 2010) Ihanteellisesti suunnittelusykli valitaan sen mittaiseksi, että mahdollisuudet reagoida yrityksen kriittisiin toimintoihin ovat optimaaliset. Liiketoiminnan suunnittelun tiheytyminen mahdollistaa lisäksi taloudellisen tilanteen tarkastelun säännöllisin väliajoin, suunnittelun tiiviimmän kytkemisen strategiaan sekä lisääntyneen dialogin ja keskustelun ylemmän johdon ja keskijohdon välille. (Lorain 2010) Rullaavat ennusteet ovat myös yleensä suhteellisen tarkkoja, sillä niitä päivitetään sekä viimeaikaisen taloudellisen kehityksen (trendien), asiakaskysynnän sekä liiketoiminnan toteumatiedon perusteella (Hope & Fraser 2003a). Ennusteiden taso tarkentuukin, kun niitä tarkennetaan ja muokataan useita kertoja vuodessa.

Rullaavaa suunnittelua voidaan tehdä myös tarpeen mukaan, jolloin nopea reagointi erilaisiin uhkakuviin mahdollistuu ajantasaisen tiedon tukiessa päätöksentekoa. Fanningin (2000) mukaan lyhyin käytännöllinen ennustustiheys on yksi kuukausi. Kuitenkin neljännesvuoden tai puolen vuoden välein tapahtuva suunnittelu tai suunnitelmien vahvistaminen on tavallisempaa. Suunnittelutiheydellä voi olla myös kielteisiä vaikutuksia, mikäli suunnitteluprosessin vaiheet eivät ole riittävän sujuvia. Lorainin (2010) mukaan rullaavasta suunnittelusta voi muodostua kallista ja aikavievää, mikäli sitä ei automatisoida riittävän tehokkaasti. Onkin hyvin tärkeää, että rullaava suunnittelu ei toista organisaation entistä budjetointiproseduuria useita kertoja vuodessa. Vaikka näin ei tehtäisikään, saattaa työmäärä tuntua kasvaneelta alkuun. Ajan myötä, lyhyelle aikavälille kasautuvan vaivan pitäisi kuitenkin vähentyä, kun rullaavasta suunnittelusta muodostuu luonteva osa johdon kuukausittaista johtamistyötä. (Hunt 2003) Onkin tärkeää, että ennustamista ja suunnittelua ei nähdä ”erikoistapahtumana” (Lamoreaux 2011).

\subsection{Yksityiskohtaisuuden määrän pienentäminen}

Jos liiketoimintaa aletaan suunnitella rullaavasti aikaisempaa useammin, on suunnittelukierrokseen kuluvan ajan vähennyttävä merkittävästi. Suunnittelu ei voi kestää enää kuukausia, kuten budjetointi yrityksissä tyypillisesti kestää. (Akten et al. 2009) Kun tarkastellaan budjetointiin ja suunnitteluun liittyviä kehitysprojekteja, on lähes kaikissa tapauksissa yksityiskohtaisuuden määrää haluttu pienentää huomattavasti verrattuna tilinpäätökseen (Fanning 2000). Toisaalta monissa yrityksissä ajatellaan edelleen, että yksityiskohtaiset taloudelliset kannanotot kuvastavat korkeampaa tarkkuutta. Vähemmän tietoa voi kuitenkin tarkoittaa merkityksellisen tiedon korostumista, kun liiketoiminnan näkymien raportointi rajoitetaan yläriveihin. (Akten et al. 2009)

Montgomeryn (2002) mukaan ennustamisen pitää tapahtua koostetulla tasolla. Kun ennusteet laaditaan ylätasolla, saadaan jäsennelty näkymä informaatioon ja samalla monimutkaisuus ja vaivannäkö vähenevät. Oikea määrä yksityiskohtaisuutta riippuu yrityksestä. Tyypillisesti leikataan pois yksityiskohtia, jotka eivät ole suoraan yhteydessä liiketoiminnan tulokseen ja joiden viilaaminen vie turhaan johdon arvokasta aikaa (Paniccia 2008). Yksityiskohtaisuus lisää työn määrää. Kilpailun koventuessa nämä resurssit on syytä ohjata strategiatyöskentelyyn, mikä auttaa johtajia muokkaamaan ja valvomaan organisaation strategiaa kasvun ja kulujen kontrollin kannalta. (Apanaschik 2007) Pidemmällä aikavälillä on tärkeää, että suunnittelu keskittyy yrityksen kasvua ja kannattavuutta kuvaaviin tekijöihin. Näitä asioita kutsutaan liiketoiminnan tai kulujen keskeisiksi ajureiksi (Paniccia 2008).

\subsection{Paino liiketoiminnan keskeisiin ajureihin}

Rullaava ennustaminen ja rullaava suunnittelu perustuvat liiketoiminnan keskeisten ajurien hyödyntämiseen. Keskittyminen liiketoiminnan keskeisiin ajureihin auttaa hyödyntämään ennusteita nopeammin ja paremmin, esimerkiksi jättämällä talousosastolle enemmän aikaa arvoa lisäävälle analyysille ja näkemyksen luomiselle. Yritysten onkin keskeistä tunnistaa suorituskyvyn kannalta kriittiset ajurit ja keskittyä näihin lukuisten yksityiskohtien sijasta. Ennusteiden mallintaminen operatiivisten ajureiden ja parametrien kautta vanhojen lukujen päivittämisen sijasta toimii hyvänä lähtökohtana ennustamiselle (Montgomery 2002). Samalla prosessia kytketään tiiviimmin arvon tuottamisen ympärille. Ajuripohjainen ennustaminen ja suunnittelu helpottavat myös syy–seuraus-analyysiä, mikä vastaavasti edistää ennakoivaa päätöksentekoa (Apanashick 2007). Rullaavat ennusteet kytketään tavallisesti esimerkiksi tilausten määrään, myyntiin, kuluihin ja pääomakuluihin (Hope & Fraser 2003a). Toisaalta johtamiselle olennainen tieto on tärkeää korostaa, mikä tarkoittaa esimerkiksi myynnin, liikevoiton, pääoman ja henkilöstön määrän huomioimista ennustamisen ajureina (Downes 1996). Myös organisaation ulkopuolinen informaatio on ennustamisessa ja suunnittelussa keskeisen tärkeää. Esimerkiksi jättämällä huomioimatta kysynnän ulkoisia indikaattoreita liiketoiminnan suunnittelu merkityksellisellä suunnitteluhorisontilla voi olla vaikeaa. (Clarke 2007) Ulkoisen suuntautumisen lisääminen voi tarkoittaa esimerkiksi sitä, että keskitytään mittaamaan omaa suoritusta kilpailijoiden suoritusta vasten, ei vanhentunutta budjettia vasten (Bourne 2004).

\subsection{Uusien suunnittelujärjestelmien hyödyntäminen}

Jotta rullaava suunnittelu mahdollistuisi, on johdon käsiteltävä tietoa nopeammin, ja tämä tarkoittaa usein erityisen ohjelmiston hankkimista ja käyttöönottoa (Myers 2001). Ensisijainen osuus prosessin tehokkuuden kasvattamisessa onkin moderneilla tietojärjestelmillä. Vanha taulukoihin perustuva systeemi korvataan tavallisesti uusilla keskitetyillä ohjelmistoratkaisuilla. Kun toisiinsa linkitetyistä Excel-taulukoista, niiden kokoamisesta ja niihin perustuvasta informaation jakelusta päästään vähitellen eroon, kasvaa budjetointiprosessin tehokkuus. (Fanning 1999) Parhaita käytäntöjä (best practices) hyödyntävät yritykset ovatkin systemaattisesti eliminoineet taulukkolaskentaohjelmiin perustuvat mallit ja raportoinnin siirtyessään uusien teknologiaratkaisujen käyttöön (Apanaschik 2007). Integroitu systeemi auttaa myös paremman yhteyden muodostamisessa suunnitelmien laatimisen ja ylläpidon välille.

Uusilla ohjelmistoilla saavutetaan tarkempia ennusteita nopeammin ja tavallisesti myös pienemmin kustannuksin. Tämä perustuu tarkoitusta varten laadittujen ennustemallien hyödyntämiseen. Mallit perustuvat selkeisiin olettamuksiin, joita on usein muutoin hankala hallita budjetointiprosessin yhteydessä. Jos ja kun ympäristö muuttuu, myös näitä olettamuksia voidaan helposti muuttaa, jolloin ennusteita ja edelleen suunnitelmia pystytään päivittämään nopeasti. (Neely et al. 2003) Kun useimmat suunnitteluparametreista sisällytetään malleihin, ei käyttäjien tarvitse huolehtia numeroista, joihin eivät pysty vaikuttamaan. Ajuripohjaisuus ja yhtenäisten mallien käyttö takaavat, että ennusteet ja suunnitelmat ovat johdettu yhteisistä luvuista käyttäen yhteisiä malleja ja algoritmeja. (Hunt 2003) Molemmat yllä mainituista seikoista lisäävät ennustamisen ja suunnittelun läpinäkyvyyttä. Yritykset, jotka ovat optimoineet ennustamisprosessinsa onnistuneesti, pystyvät generoimaan uuden taloudellisen ennusteen jopa alle 24 tunnissa. Olennaista on, että ennusteet perustuvat tietojen päivittämiselle, jolloin ennusteita ei tarvitse laatia tyhjästä joka kerta. (Apanashick 2007) Tehokas ennustaminen tuottaa edelleen ajankohtaista tietoa suunnitelmien päivittämiseksi.

Kun prosessia automatisoidaan, on otettava huomioon myös useat ei-taloudelliset liiketoiminnan ajurit ja niihin liittyvä data. Kulujen arvioinnin ohella päälliköt arvioivat esimerkiksi myynnin tehokkuutta, kapasiteetin käyttöä tai asiakaspalvelun tehokkuutta saadakseen selville lopulliset kulunsa. Useasti tämä mallintaminen tapahtuu edelleen paikallisesti eri ihmisten taulukkolaskentaohjelmissa. Suuri osa keskeisestä operatiivisesta datasta, jonka perusteella säädellään tulevaisuuden resurssitarpeita ja edelleen liikevaihdon luomista on siis lukittuna yksittäisten ihmisten hallinnoimiin taulukoihin. (Barret & Hope 2006) Nämä pitäisikin pystyä integroimaan mahdollisimman saumattomasti keskitettyyn suunnittelumalliin tai suunnittelutyökaluun, jolloin suunnittelua tekevät osapuolet pystyvät mahdollisimman helposti ja nopeasti päivittämään keskeiset operatiiviset ajurit ennusteiden tueksi. Ainoastaan tällöin on mahdollista konsolidoida koko organisaatiota koskevia ennusteita riittävän nopeasti. (Barret & Hope 2006) Edellä kuvattu ilmiö selittää myös osaltaan, kuinka taulukkolaskentaa käyttävät organisaatiot voivat toisinaan tehdä ennustekierroksia lähes yhtä nopeasti kuin valmisohjelmistoja hyödyntävät organisaatiot. Joka tapauksessa niin sisäisistä kuin ulkoisistakin tietolähteistä saatavan tiedon syöttämisen automatisointi budjetointi- ja ennustamisprosessin työvaiheissa on oleellista, sillä ilman automaattista syöttöä talouden suunnittelijat käyttävät merkittävästi aikaa tiedon keruuseen, validointiin ja täsmentämiseen.

Uusilla ohjelmistoratkaisuilla on myös mahdollista lisätä osallistumista. Ennustamisen vieminen operatiiviselle tasolle on osoittautunut parhaaksi menetelmäksi tarkkojen ja luotettavien ennusteiden saamiseksi. Tämän on mahdollistanut edistynyt teknologia ja esimerkiksi selainpohjaiset ratkaisut, minkä ansiosta eri ratkaisut skaalautuvat jopa sadoille tai tuhansille loppukäyttäjille, antaen halutuille henkilöille mahdollisuudet ennustamiseen ja suunnitteluun. Tämä mahdollistaa oman vastuualueen liiketoiminnan ennustamisen myös operatiivisen tason johtajille. Tarkkuus lisääntyy operatiivisen tason johtajien päästessä ennustamaan asioita, joihin he pystyvät vaikuttamaan ja joista keskustelevat päivittäin. (Hunt 2003) Hyvänä tavoitteena on varmistaa, että johto on sinut suunnittelutyökalun kanssa ja ymmärtää oman osuutensa suunnittelun kokonaisuudessa koko organisaation tasolla. Nykyaikaiset työkalut tarjoavat erilaisia ominaisuuksia, kuten työvuorojen suunnittelun tai tuotesuunnittelua, jotka ovat suunniteltu sitä varten, että johtajat huomioisivat suunnittelutehtävät säännöllisesti, sen sijaan että he suunnittelisivat kerran tai kahdesti vuodessa. (Leahy 2006)

\section{Rullaava suunnittelu johtamisjärjestelmän kannalta}

\subsection{Taktisen suunnittelun kytkeminen strategiaprosessiin}

Strategisen suunnitteluprosessin kytkeminen taktiseen eli lyhyen aikavälin suunnitteluun on oleellista strategian tehokkaassa toimeenpanossa (Apanaschik 2007). Johtavat yritykset ymmärtävät, että taloudellinen suorituskyky muodostuu lopulta hyvien kilpailustrategioiden kehittämisestä ja toteuttamisesta (Neely et al. 2003). Kun strategian toteuttamisesta muodostetaan yksi ydinkompetensseista, on mahdollista saavuttaa kilpailuetua muihin nähden (Cokins 2008). Rullaavan suunnittelun yhteydessä on luontevaa tarkastella lyhyen aikavälin ennustetta suhteessa strategiassa asetettuihin pitkän tähtäimen tavoitteisiin. Parhaassa mahdollisessa tapauksessa säännöllinen tarkastelu paljastaa taloudellisia kuiluja ennen niiden toteutumista, mahdollistaa reagoinnin, toimintasuunnitelman päivittämisen ja edelleen tavoitteissa pysymisen. Rullaavan suunnittelun avulla saadaan parempi näkyvyys ja jatkuvasti palautetta, minkä perusteella strategiaa voidaan muokata. Näin ollen myös oppiminen on yksi sen ulottuvuuksista (Lorain 2010).

Lyhyen aikavälin ja strategisen suunnittelun kytkeminen toisiinsa on ensisijaisesti prosessiin liittyvä haaste, ei niinkään esimerkiksi ohjelmistoihin liittyvä (Rickards 2008). Onkin mietittävä, miten strategisen suunnitelman käsitteet pystytään muuttamaan spesifeihin taloudellisiin mittareihin (Montgomery 2002). Kun lyhytaikaisia taloudellisia mittareita täydennetään pidemmän aikavälin mittareilla, on strategisten tavoitteiden saavuttamista helpompi seurata (Fanning 1999). Prosessin pitäisi kokonaisuutena kannustaa johtajia ja muita päätöksentekijöitä ajattelemaan budjetin yksityiskohtien ja lyhytaikaisten tavoitteiden ylitse ja keskittymään siihen, mihin suuntaan liiketoiminta on kulkeutumassa (Montgomery 2002). Johtavat yritykset käyttävät ennusteita tehdäkseen parempia arvioita siitä, onko strategian mukainen suunta toteutumassa sen sijaan, että vain tarkastelisivat tilannetta suhteessa vuosisuunnitelmaan (Barret & Hope 2006).

\subsection{Tavoiteasetanta rullaavassa järjestelmässä}

Tavoiteasetanta, suorituskyvyn mittaaminen ja palkitseminen ovat johtamisjärjestelmän keskeisiä komponentteja. Perinteisesti taloudellinen vuosibudjetti on toiminut pääasiallisena vertailukohteena suorituskyvylle. Barsky et al. (1999) huomauttavat, että talouden vauhdin kiihtyessä globaalissa liiketoimintaympäristössä on tärkeää siirtää katseet mennyttä kuvaavien taloudellisten indikaattoreiden sijasta pitkän tähtäimen arvonluontiin keskittyviin mittareihin. Lorainin (2010) mukaan rullaavat ennusteet eivät yksinään riitä korvaamaan perinteistä budjetointiprosessia. Tämä johtuu siitä, että ne eivät huomioi suorituskyvyn mittaamista ja motivointia, jotka ovat oleellisia työn tehokkuuden johtamisessa. Rullaavat ennusteet yhdessä rullaavan suunnittelun kanssa sen sijaan kattavat myös tavoiteasetannan funktion. Joka tapauksessa, siirryttäessä perinteisestä budjetoinnista täysin rullaavaan suunnitteluprosessiin on selvitettävä, millä tavoin tavoiteasetanta, suorituskyvyn mittaaminen ja palkitseminen tulee muuttumaan. On tärkeää tunnistaa, millä tavoin perinteinen budjetointi on kytketty kuhunkin johtamisjärjestelmän komponenttiin, ja osoitettava vastaava linkki uudessa suunnitteluprosessissa.

Tavoiteasetannan ja seurannan prosessilla on rullaavassa suunnittelussa olennainen merkitys (Åkerberg 2006). Kun ennusteita päivitetään rullaavasti, on tavoitteiden päivittäminen tarpeen vaatiessa samalla luontevaa. Samalla, kun suunnitelmat päivitetään ja vahvistetaan, asetetaan uudet tavoitteet. Toisaalta kirjallisuudessa painotetaan myös pitkäaikaisten tavoitteiden painoarvoa. Osaksi huomiot liittyvät muuttuvien tavoitteiden ja palkitsemisen suhteeseen. Akten et al. (2009) toteavat, että uusien tavoitteiden neuvotteleminen ja kannustimien uudelleen asettaminen saattaa viedä huomiota pois liiketoiminnan suunnittelusta ja politisoida suunnitteluprosessia. Åkerbergin (2006) mukaan merkittävin osa tavoitteista tulisikin asettaa strategiatyöskentelyn yhteydessä, mikä on myös yksi syy, minkä takia strategia on kytkettävä lyhyen aikavälin suunnitteluun. Tätä tukee tutkimus, jonka mukaan budjetoinnista luopumisen jälkeen tavoiteasetannan pitäisi perustua strategiaan ja pitkäaikaisiin päämääriin, fokus tulisi yksityiskohtien sijasta olla kokonaiskuvassa, mahdollisuuksissa ja joustavuudessa, ennemmin kuin rajoitteissa (Østergren & Stensaker 2011).

\subsection{Palkitsemiskäytännöt}

Edellä käsiteltiin tavoiteasetantaan liittyviä haasteita. Rullaavan suunnittelun yhteydessä myös palkitsemiskäytäntöjä on syytä tarkastella uusiksi (Åkerberg 2006). Monissa organisaatioissa rullaavan suunnittelun käyttöönotto koetaan haasteelliseksi, sillä palkitsemisprosessit ovet yleensä tiiviisti budjettiin kytkettyjä. Apanaschik (2007) toteaakin, että tämän seurauksena osa yrityksistä ei ole myöskään pystynyt kokonaan irrottautumaan budjetoinnista, vaikka rullaava suunnitteluprosessi olisi jo käytössä. Lazeren (1998) mukaan mihin tahansa liiketoiminnan suunnittelutyökaluun, on kytkettävä suorituksen arviointi ja palkitseminen, jotta siitä saataisiin täysi hyöty irti. Buttonwood Groupin Lawrence Servenin mukaan ihmisten motivointi saattaakin olla haastavaa, ellei palkitsemista oteta mukaan suunnitteluprosessin uudistamiseen. Ongelmana on, että talousosasto näkee palkitsemisen yleensä tiukasti henkilöstöosaston alaisuuteen kuuluvana asiana. Servenin mukaan ratkaisuna tähän toimii henkilöstöosaston ottaminen mukaan kehitysprojektiin. Tämän lisäksi hän mainitsee, että tyytyväisyys uudistamisprojekteissa on ollut yleisesti korkeampi, kun palkitseminen on otettu huomioon. (Moriarty 2011)

Perinteisen budjetoinnin palkitsemiseen liittyvät ongelmat eivät ratkea itsestään siirtymällä rullaavaan suunnitteluun. Palkitsemisjärjestelmän muuttaminen on kuitenkin haastavaa. Lorainin (2010) tutkimuksessa kaikki vastaajat mainitsivat, että suorituksen arviointi ja palkitseminen perustuvat todellisuudessa edelleen saavutettujen tulosten ja budjetin vertailuun. Tällöin perinteisen budjetoinnin palkitsemiseen liittyviä ongelmia ei ole poistettu. Eettisesti toimivassa yrityksessä tulisikin pyrkiä palkitsemiseen aidoista suorituksista, ei kyvystä ”budjetoida oikein”. (Åkerberg 2006) Yksi kirjallisuudessa ehdotettu ratkaisu palkitsemiseen liittyviin ongelmiin on sekä ei-taloudellisten että pitkän tähtäimen suoritusmittareiden huomiointi palkitsemisessa (Hansen et al. 2003). Koska tämäntyyppiset mittarit eivät ole niin herkkiä talouden makroilmiöille (esimerkiksi lama, bruttokansantuotteen lasku tai inflaatio), motivoituvat johtajat huomioimaan laajemmin erilaisia skenaarioita liiketoiminnan hyväksi (Akten et al. 2009).  Lisäksi tämä auttaa ihmisiä tekemään parhaansa ja pitämään tuntosarvet ylhäällä ympäristön ja kilpailijoiden suhteen ja heidän tekemisiinsä – ei ainoastaan peilaamaan omaa liiketoimintaa yrityksen budjettiin (Daum et al. 2005).

Yksi vaihtoehto on ottaa käyttöön suhteelliset suorituskykymittarit. Tällöin ihmiset pitäisi saada ymmärtämään, että tavoitteen näkeminen on tärkeää, mutta absoluuttisia numeroita ei välttämättä tarvitse tietää etukäteen (Daum et al. 2005). Siirryttäessä suhteelliseen tavoiteasetantaan, on johtajien hyväksyttävä se, että tavoitetta asetettaessa absoluuttisia numeroita ei ole tiedossa. Tämä edellyttää ajatusmallien muutosta, mikäli organisaatiossa on totuttu esimerkiksi perinteiseen budjetointiin. Tavoitteet voivat perustua osin yrityksen ulkopuolisten komponenttien kehitykselle, esimerkiksi kilpailijoiden suoritukseen peilaamiseen (Daum et al. 2005), markkinaosuuden suhteelliseen kasvuun tai asiakastyytyväisyyteen (Akten et al. 2009). Asiat, joihin johtajat eivät itse voi vaikuttaa on syytä karsia pois. Hansen et al. (2003) huomauttavat, että useimmilla yrityksillä ei välttämättä ole kuitenkaan riittävän hyvää suhteellista dataa kilpailevien yritysten suorituksesta. Tämä saattaa johtua esimerkiksi alalla vallitsevasta kovasta kilpailusta, jolloin yritykset eivät jaa tietoa yhtä avoimesti. Toisaalta juuri kovasti kilpailuilla aloilla suhteellinen suorituskyvyn mittaus olisi potentiaalisesti kaikista hyödyllisin. Toinen huomionarvoinen seikka on se, että vaikka teoriassa onkin ehdotettu, missä suhteellista suorituskyvyn mittausta (RPE) kannattaisi käyttää, on empiirinen näyttö suhteellisen suorituskykymittariston käyttöönottamisesta vielä vähäistä. (Hansen et al. 2003)

Eccles (1991) ehdotti, että suorituskyvyn arviointimallissa on otettava huomioon erityisesti tärkeimmät suorituskyvyn mittarit yrityksen strategian ja pitkän tähtäimen taloudellisen menestyksen kannalta. Nämä ovat usein ei-taloudellisia mittareita, joita ei välttämättä huomioida lainkaan palkitsemisessa, vaikka yrityksissä muuten esimerkiksi laatua ja markkinaosuutta seurattaisiinkin. (Eccles 1991) Työntekijöidenkin pitäisi kiinnostua tämäntyyppisistä palkitsemismalleista ymmärtäessään, miten palkitseminen on kytketty liiketoiminnan arvon kasvuun (McLemore 1996). Ei-taloudellisia mittareita ovat muun muassa strategian toteuttaminen, strategian laatu, innovatiivisuus, kyky houkutella lahjakasta työvoimaa, markkinaosuus ja keskeisten prosessien laatu. Näitä mittareita on mahdollisuus seurata esimerkiksi tuloskorteilla (Barsky & Bremser 1999).

Palkitsemiseen voidaan liittää myös työntekijöiden itsearviointi, sisältäen ei-lukumääräisten saavutusten arviointia. Menetelmän käyttö osoittaa luottamusta työntekijöitä kohtaan, vaikkakin arvioinnit tarkistutettaisiin esimiehillä. Se myös siirtää työn arvioinnin painopistettä työn laatuun ja tietoon siitä, miten työntekijä on pystynyt kehittämään yrityksen toimintaa ja vaikuttanut tuloksen muodostumiseen. Näin voidaan edistää kulttuuria, jossa etusijalla on yhden osaston tai henkilön edun sijaan koko yrityksen etu kokonaisuutena. (McGee 2003) Erityisen tärkeää on, että työntekijät pystyvät vaikuttamaan itse niihin asioihin, joihin palkitseminen sidotaan. Keskijohdolle on hyvä antaa vapauksia käyttää omaa harkintakykyänsä suunnitelmien yksityiskohtien viilaamisessa saavuttaakseen asetettuja tavoitteita. Budjetin rajoitusten noudattamisen sijasta on motivoivampaa, että johtajat voivat tehdä päätöksiä joustavasti tiettyjen strategiaan sidottujen toimintarajojen puitteissa ja käyttää yrityksen resursseja asetettujen tavoitteiden saavuttamiseen. Fokuksen ei siis pitäisi olla budjetin asettamiin ehtoihin mukautumisessa, vaan yritysten tavoitteiden saavuttamisessa käyttäen saatavilla olevia resursseja tehokkaasti. (Etim & Agara 2011)

\subsection{Suhde tuloskorttijärjestelmään}

Tuloskortti (balanced scorecard) on Robert Kaplanin ja David Nortonin kehittämä johtamismenetelmä (Kaplan & Norton 1996). Se on suunniteltu strategiseksi johtamisjärjestelmäksi ja sen keskeisenä tavoitteena on auttaa organisaatiota strategisten tavoitteiden kääntämisessä merkityksellisiksi suorituskykymittareiksi. Clarken (2007) mukaan rullaavan suunnittelun suosiolle on kaksi pääsyytä: Se auttaa keskittämään suunnittelijoiden huomion vuosittaisen maaliviivan taakse ja kannustaa johtoa luovempaan ajatteluun sekä riskien ja mahdollisuuksien tarkkailuun niin nykyhetken, kuin tulevaisuudenkin suhteen. Toiseksi, rullaavan suunnittelun yhteyteen on luontevaa liittää eri suorituskyvyn mittaamisen järjestelmiä, kuten tuloskortit.  Kun ennusteiden ja suunnitelmien ajurit määritetään esimerkiksi tuloskorttien tavoitteiden pohjalta, niin tavoitteiden toteutumista on helpompi seurata pala kerrallaan. (Clarke 2007) Kun sekä taloudelliset että ei-taloudelliset mittarit otetaan huomioon, pystytään seuraamaan strategian toteutumista koko organisaation laajuudelta, ja kertomaan, mikäli tavoitteita saavutetaan tai ei saavuteta (Barsky et al. 1999).

Bogsnesin (2010) mukaan kirjallisuudessa on nostettu esiin huoli budjettien ja tuloskorttien yhtäaikaisesta käytöstä organisaatioissa. Hänen mukaansa näillä kahdella eri ohjausjärjestelmällä saattaa olla vastakkaisia viestejä johdon käyttäytymistä ajatellen. Jos eri menetelmiä käytetään rinnakkain, saattaa vahvasti hierarkkisissa organisaatioissa olla haastavaa ohittaa budjetointiin ankkuroidut muodolliset vastuu- ja auktoriteettikäytännöt. Budjetti ja tuloskortti ovatkin monilla tavoin kaksi hyvin erilaista systeemiä, sillä niiden tavoitteet ja luonne on erilainen.  (Bourmistrov & Øestegren Kaarbøe 2011) Jos budjetin ja esimerkiksi tuloskortin ohjausvaikutus on erilainen, saattaa päälliköiden olla vaikeaa päättää kumpaa pitäisi seurata. Tällöin ylemmän johdon painottama menetelmä saa vahvemman painoarvon. Mikäli budjetointia painotetaan, jää tuloskorttien merkitys vähäisemmäksi, ja päinvastoin. Näin ollen ne eivät koskaan ole aivan tasavertaisia. (Bourmistrov & Øestegren Kaarbøe 2011) Samalla tavalla on varmistettava, että tuloskorttien ja rullaavan suunnittelun ohjausvaikutukset eivät ole keskenään ristiriidassa.

\subsection{Rullaava suunnittelu ja organisaatiokulttuurin muutos}

Ehyttä ja toimivaa rullaavaa ennustamisprosessia, johon käyttäjät pystyvät luottamaan, ei rakenneta hetkessä. Rullaavuuden omaksumiseen organisaatiossa tarvitaan kulttuurin ja ajattelutapojen muutosta. Rullaava suunnittelu suuntaa yritysten katseen tulevaisuuden tuloksen johtamiseen, ei menneen suorituksen selittelyyn. Menneen tuloksen tarkastelu lisää vain vähän arvoa. Tärkeämpää on pyrkiä ennustamaan vaihteluja ennen niiden realisoitumista ja keskittyä toimenpiteisiin, jotka edistävät arvon luontia. Useimmat näistä toimenpiteistä ovat ei-taloudellisia, mikä korostaa perinteisen budjetoinnin rajallisuutta. (Bourne 2004) Johtavat yritykset eivät juuri selittele menneisyyttä, vaan keskittyvät tulevaisuuden tuloksen johtamiseen. Tämä tapahtuu ennustamalla todennäköiset vaihtelut ennen niiden tapahtumista. Johtamiseen käytetään tuoreinta ennustetta toteutuneen tuloksen seuraamisen sijasta. (Neely et al. 2003)

Akten et al. (2009) havaitsivat tutkimuksessaan, että yrityksissä pystytään tunnistamaan poikkeumat suhteessa tavoiteasetantaan rullaavan ennustamisen yhteydessä. Tämän jälkeen selvitetään keskustellen eri osapuolten kanssa mistä suorituskyvyn erot johtuvat, mikä luo tietä tarvittaville päätöksille. Yrityksille, jotka eivät ole tottuneet tämäntyyppiseen yhteistyöhön suunnittelun yhteydessä, tämä edustaa suurta kulttuurin muutosta, sillä johtajien vastuu toimistaan korostuu ja koko organisaation on kollektiivisesti sopeuduttava nopeasti muuttuvaan makroekonomiseen ympäristöön. (Akten et al. 2009) Rullaavan suunnitteluprosessin onkin rohkaistava dialogiin ja keskusteluun läpi organisaation. Tähän liittyy myös se, että rullaavan suunnitteluprosessin katsotaan tavallisesti lisäävän suunnittelun läpinäkyvyyttä ja siten vastuiden selkeytymistä. (Lorain 2010) Myös Apanaschik (2007) korostaa organisaation yhteistyön lisääntymisen merkitystä. Optimaalinen ennustamis- ja suunnitteluprosessi vaatii vahvaa vuorovaikutusta talouden, operatiivisen puolen ja liiketoimintayksiköiden johtajien välillä. Talousosastoilla on keskeinen asema säännöllisten taloudellisten suunnitelmien ja ennustamisen hyödyntämisessä ymmärtääkseen paremmin liiketoimintaa ja tehdäkseen suunnittelun osaksi yrityskulttuuria. Toisaalta on muistettava, että tehokas ja pitkällä tähtäimellä arvokas taloudellinen suunnittelu pohjautuu siihen, että se tuo lisäarvoa liiketoimintayksiköille. (Apanaschik 2007)

\subsection{Johtamisjärjestelmän muokkaaminen}

Viime vuosien aikana yritykset ovat investoineet ennustamis- ja suunnitteluprosessien kehittämiseen. Tällaiset investoinnit ovat yleensä isoja, jopa useita miljoonia euroja. Tämä pitää tyypillisesti sisällään konsultoinnin sekä yrityksen oman työvoiman ajan sekä modernin ohjelmiston, joka kokoaa datan koko organisaatiosta. Tällainen uudistamisprojekti vaatii kärsivällisyyttä, intensiivistä kommunikaatiota työntekijöiden kanssa, uusiin tiedonsyöttövälineisiin ja rajapintoihin investointia ja yrityksen talousosaston kehittymistä datan kerääjistä ja levittäjistä liiketoiminnan analyytikoiksi. Joidenkin arvioiden mukaan jopa puolet yrityksistä luovuttaa näin isojen projektien kuluessa ja jättävät jo aloitetun kehittämistyön kesken.  (Banham 2000) Tyypillisesti useimmat näistä projekteista kohtaavat saman haasteen: teoriassa ne näyttävät hyvältä, mutta suunnitelmien implementoinnissa törmätään vaikeuksiin. Usein hankkeet kompuroivat keskitetyn päätöksenteon järkkymättömään asemaan, jäykkiin suoritussopimuksiin sekä vuosittaisen budjetoinnin immuniteettiin. (Hope et al. 2007)

Johdon retoriikan muuntaminen toimivaksi käytännöksi onkin suuri haaste. Reagointikyky, alaisten valtuuttaminen (empowerment), innovaatio, operatiivinen erinomaisuus tai asiakaskeskeisyys ei ole helposti saavutettavissa, mikäli johtamisprosesseja ei päivitetä vastaamaan nykyajan vaatimuksia. Jäykät strategiat estävät nopean reagoinnin, organisaatiorakenne rajoittaa haastetta ja kehitystä etsiviä johtajia, väärin asetetut tavoitteet saattavat heikentää asiakkaan huomiointia ja lyhyen tähtäimen tavoitteet estävät pitkän tähtäimen arvonluontia. (Hope et al. 2007) Tämän takia, kun puhutaan niin kokonaisvaltaisesta asiasta kuin suunnitteluprosessin kehittämisestä, on johtamisjärjestelmä ehdottomasti huomioitava muutoksessa. Ideaalitilanteessa, ennustamis- ja suunnitteluprosessit muodostuvat niin luontevaksi osaksi operatiivista johtamista, että ennusteiden ja suunnitelmien tuottaminen on käytännössä vain operatiivisen ja normaalin työskentelyn ohessa syntyvä sivutuote. (Hunt 2003)

Pahimmillaan rullaavan suunnitteluprosessin käyttöönotto saattaa johtaa siihen, että entinen budjetointiprosessi toistetaan monta kertaa vuodessa. Tämä ei tietenkään ole tarkoitus, eikä se ole hyväksi prosessin luotettavuudelle, osallistujien motivaatiolle ja yleiselle jaksamiselle. Jotta tähän ei ajauduttaisi, on syytä pohtia mitä rullaavalla prosessilla halutaan saavuttaa ja miksi. (Åkerberg 2006) Kun suunnitteluvaihe on valmis, tulisi ylimmän johdon kommunikoida selkeät, konkreettiset ja läpinäkyvät strategisen tason tavoitteet (Lorain 2010). Hyvin toimiva johtamisjärjestelmä on herkkä ja monimutkainen systeemi. Sen jokaisen osan on ohjattava toimintaa samaan suuntaan. Vain silloin voi organisaatio minimoida sisäiset konfliktinsa ja maksimoida oman potentiaalinsa. Rakenne on suunniteltava holistisista lähtökohdista; johtajuusongelmat ennen johtamisprosessia ja johtamisprosessit ennen uusia järjestelmiä ja työkaluja, sillä kaiken on oltava linjassa keskenään ja tuettava uusia malleja ja käytäntöjä. (Hope et al. 2007)



You can create PDF files out of practically anything.
In Windows, you can download PrimoPDF or CutePDF (or some such) and install a
printing driver so that you can print directly to PDF files from any
application. There are also tools that allow you to upload documents in common
file formats and convert them to the PDF format.
If you have PS or EPS files, you can use the tools \texttt{ps2pdf} or
\texttt{epspdf} to convert your PS and EPS files to PDF\@.

% Comment: If your sentence ends in a capital letter, like here, you should
% write \@ before the period; otherwise LaTeX will assume that this is not
% really an end of the sentence and will not put a large enough space after the
% period. That is, LaTeX assumes that you are (for example), enumerating using
% capital roman numerals, like I. do something, II. do something else. In this
% case, the periods do not end the sentence.

% Similarly, if you do need a normal space after a period (instead of
% the longer sentence separator), use \  (backslash and space) after the
% period. Like so: a.\ first item, b.\ second item.

Furthermore, most newer editor programs allow you to save directly to the PDF
format. For vector editing, you could try Inkscape, which is a new open source
WYSIWYG vector editor that allows you to save directly to PDF\@.
For graphs, either export/print your graphs from OpenOffice Calc/Microsoft
Excel to PDF format, and then add them; or use \texttt{gnuplot}, which can
create PDF files directly (at least the new versions can).
The terminal type is \emph{pdf}, so the first line of your plot file should be
something like \texttt{set term pdf \ldots}.

To get the most professional-looking graphics, you can encode them using the
TikZ package (TikZ is a frontend for the PGF graphics formatting system).
You can create practically any kind of technical images with TikZ, but it has a
rather steep learning curve. Locate the manual (\texttt{pgfmanual.pdf}) from
your \LaTeX\ distribution and check it out. An example of TikZ-generated
graphics is shown in Figure~\ref{fig:page-merge}.

\begin{figure}[ht]
  \begin{center}
    \input{example_page-merge.tex}
    \caption{Example of a multiversion database page merge. This figure has
    been taken from the PhD thesis of Haapasalo~\cite{HaapasaloThesis}.}
    \label{fig:page-merge}
  \end{center}
\end{figure}

Another example of graphics created with TikZ is shown in
Figure~\ref{fig:tikz-examples}.
These show how graphs can be drawn and labeled.
You can consult the example images and the PGF manual for more examples of what
kinds figures you can draw with TikZ.

% These definitions are only used in the example images; you will not
% need them for your thesis...
\newlength{\graphdotsize}
\setlength{\graphdotsize}{1.7pt}
\newlength{\graphgridsize}
\setlength{\graphgridsize}{1.2em}
\begin{figure}[ht]
\begin{center}
\subfigure[Examples of obstruction graphs for the Ferry Problem]{
  \input{example_obstruction-grouped.tex}
}
\subfigure[Examples of star graphs]{
  \input{example_general-star-graphs.tex}
}
\caption{Examples of graphs draw with TikZ. These figures have been taken from a
course report for the graph theory course~\cite{FerryProblem}.}
\label{fig:tikz-examples}
\end{center}
\end{figure}



% \input{4methods.tex}

\chapter{Methods}
\label{chapter:methods}

You have now stated your problem, and you are ready to do something
about it!  \emph{How} are you going to do that? What methods do you
use?  You also need to review existing literature to justify your
choices, meaning that why you have chosen the method to be applied in
your work.

% An example of a traditional LaTeX table
% ------------------------------------------------------------------
% A note on underfull/overfull table cells and tables:
% ------------------------------------------------------------------
% In professional typography, the width of the text in a page is always a lot
% less than the width of the page. If you are accustomed to the (too wide) text
% areas used in Microsoft Word's standard documents, the width of the text in
% this thesis layout may suprise you. However, text in a book needs wide
% margins. Narrow text is easier to read and looks nicer. Longer lines are
% hard to read, because the start of the next line is harder to locate when
% moving from line to the next.
% However, tables that are in the middle of the text often would require a wider
% area. By default, LaTeX will complain if you create too wide tables with
% ``overfull'' error messages, and the table will not be positioned properly
% (not centered). If at all possible, try to make the table narrow enough so
% that it fits to the same space as the text (total width = \textwidth).
% If you do need more space, you can either
% 1) ignore the LaTeX warnings
% 2) use the textpos-package to manually position the table (read the package
%    documentation)
% 3) if you have the table as a PDF document (of correct size, A4), you can use
%    the pdfpages package to include the page. This overrides the margin
%    settings for this page and LaTeX will not complain.
% ------------------------------------------------------------------
% Another note:
% ------------------------------------------------------------------
% If your table fits to \textwidth, but the cells are so narrow that the text
% in p{..}-formatted cells does not flow nicely (you get underfull warnings
% because LaTeX tries to justify the text in the cells) you can manually set
% the text to unjustified by using the \raggedright command for each cell
% that you do not want to be justified (see the example below). \raggedleft
% is also possible, of course...
% ------------------------------------------------------------------
% If you need to have linefeeds (\\) inside a cell, you must create a new
% paragraph-formatting environment inside the cell. Most common ones are
% the minipage-environment and the \parbox command (see LaTeX documentation
% for details; or just google for ``LaTeX minipage'' and ``LaTeX parbox'').
\begin{table}
\begin{tabular}{|p{2cm}|p{3.8cm}|p{4.5cm}|p{1.1cm}|}
% Alignment of sells: l=left, c=center, r=right.
% If you want wrapping lines, use p{width} exact cell widths.
% If you want vertical lines between columns, write | above between the letters
% Horizontal lines are generated with the \hline command:
\hline % The line on top of the table
\textbf{Code} & \textbf{Name} & \textbf{Methods} & \textbf{Area} \\
\hline
% Place a & between the columns
% In the end of the line, use two backslashes \\ to break the line,
% then place a \hline to make a horizontal line below the row
T-110.6130 & Systems Engineering for Data Communications
    Software & \raggedright Computer simulations, mathematical modeling,
  experimental research, data analysis, and network service business
  research methods, (agile method) & T-110 \\
\hline
\multicolumn{2}{|p{6.25cm}|}{Mat-2.3170 Simulation (here is an example of
 multicolumn for tables)}& Details of how to build simulations & T-110 \\
% The multicolumn command takes the following 3 arguments:
% the number of cells to merge, the cell formatting for the new cell, and the
% contents of the cell
\hline
S-38.3184 & Network Traffic Measurements and Analysis
& \raggedright How to measure and analyse network
  traffic & T-110 \\ \hline
\end{tabular} % for really simple tables, you can just use tabular
% You can place the caption either below (like here) or above the table
\caption{Research methodology courses}
% Place the label just after the caption to make the link work
\label{table:courses}
\end{table} % table makes a floating object with a title

If you have not yet done any (real) metholodogical courses (but chosen
introduction courses of different areas that are listed in the
methodological courses list), now is the time to do so or at least
check through material of suitable methodological courses. Good
methodologial courses that consentrates especially to methods are
presented in Table~\ref{table:courses}. Remember to explain the
content of the tables (as with figures). In the table, the last column
gives the research area where the methods are often used. Here we used
table to give an example of tables. Abbreviations and Acronyms is also
a long table. The difference is that longtables can continue to next
page.



% \input{5implementation.tex}

\chapter{Implementation}
\label{chapter:implementation}

You have now explained how you are going to tackle your problem.
Go do that now! Come back when the problem is solved!

Now, how did you solve the problem?
Explain how you implemented your solution, be it a software component, a
custom-made FPGA, a fried jelly bean, or whatever.
Describe the problems you encountered with your implementation work.



% \input{6evaluation.tex}

\chapter{Evaluation}
\label{chapter:evaluation}

You have done your work, but that's\footnote{By the way, do \emph{not} use
shorthands like this in your text! It is not professional! Always write out all
the words: ``that is''.} not enough.

You also need to evaluate how well your implementation works.  The
nature of the evaluation depends on your problem, your method, and
your implementation that are all described in the thesis before this
chapter.  If you have created a program for exact-text matching, then
you measure how long it takes for your implementation to search for
different patterns, and compare it against the implementation that was
used before.  If you have designed a process for managing software
projects, you perhaps interview people working with a waterfall-style
management process, have them adapt your management process, and
interview them again after they have worked with your process for some
time. See what's changed.

The important thing is that you can evaluate your success somehow.
Remember that you do not have to succeed in making something spectacular; a
total implementation failure may still give grounds for a very good master's
thesis---if you can analyze what went wrong and what should have been done.




% \input{7discussion.tex}

\chapter{Discussion}
\label{chapter:discussion}

At this point, you will have some insightful thoughts on your implementation
and you may have ideas on what could be done in the future.
This chapter is a good place to discuss your thesis as a whole and to show your
professor that you have really understood some non-trivial aspects of the
methods you used\ldots



% \input{8conclusions.tex}

\chapter{Conclusions}
\label{chapter:conclusions}

Time to wrap it up!
Write down the most important findings from your work.
Like the introduction, this chapter is not very long.
Two to four pages might be a good limit.



% Load the bibliographic references
% ------------------------------------------------------------------
% You can use several .bib files:
% \bibliography{thesis_sources,ietf_sources}
\bibliography{ref}


% Appendices go here
% ------------------------------------------------------------------
% If you do not have appendices, comment out the following lines
\appendix
% \input{appendices.tex}

\chapter{First appendix}
\label{chapter:first-appendix}

This is the first appendix. You could put some test images or verbose data in an
appendix, if there is too much data to fit in the actual text nicely.

For now, the Aalto logo variants are shown in Figure~\ref{fig:aaltologo}.

\begin{figure}
\begin{center}
\subfigure[In English]{\includegraphics[width=.8\textwidth]{aalto-logo-en}}
\subfigure[Suomeksi]{\includegraphics[width=.8\textwidth]{aalto-logo-fi}}
\subfigure[Pä svenska]{\includegraphics[width=.8\textwidth]{aalto-logo-se}}
\caption{Aalto logo variants}
\label{fig:aaltologo}
\end{center}
\end{figure}


% End of document!
% ------------------------------------------------------------------
% The LastPage package automatically places a label on the last page.
% That works better than placing a label here manually, because the
% label might not go to the actual last page, if LaTeX needs to place
% floats (that is, figures, tables, and such) to the end of the
% document.
\end{document}
